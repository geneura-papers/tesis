% **************************************************************************************************************
% A Classic Thesis Style
% An Homage to The Elements of Typographic Style
%
% Copyright (C) 2007 Andr� Miede http://www.miede.de
%
% If you like the style then I would appreciate a postcard. My address
% can be found in the file ClassicThesis.pdf. A collection of the
% postcards I received so far is available online at
% http://postcards.miede.de
%
% License:
% This program is free software; you can redistribute it and/or modify
% it under the terms of the GNU General Public License as published by
% the Free Software Foundation; either version 2 of the License, or
% (at your option) any later version.
%
% This program is distributed in the hope that it will be useful,
% but WITHOUT ANY WARRANTY; without even the implied warranty of
% MERCHANTABILITY or FITNESS FOR A PARTICULAR PURPOSE.  See the
% GNU General Public License for more details.
%
% You should have received a copy of the GNU General Public License
% along with this program; see the file COPYING.  If not, write to
% the Free Software Foundation, Inc., 59 Temple Place - Suite 330,
% Boston, MA 02111-1307, USA.
%
% **************************************************************************************************************
% Note:
%    * You must not use "u etc. in strings/commands that will be spaced out (use \"u or real umlauts instead)
%    * Chapters must be marked with the \myChapter{Foo} command (sorry for the inconvenience at this point)
%    * New enumeration (small caps): \begin{aenumerate} \end{aenumerate}
%    * For margin notes: \graffito{}
%    * Do not use bold fonts in this style, it is designed around them
%    * Use tables as in the examples
%    * See classicthesis-ldpkg.sty for useful commands
% **************************************************************************************************************
% To Do:
%    * remove obsolete KOMA options and use \KOMAoptions instead
%    * support a List of Listings that looks like the other lists
% **************************************************************************************************************
%\documentclass[twoside,titlepage,fleqn,BCOR5mm]{scrbook}%,BCOR5mm

% Comentado por JCC 
\documentclass[10pt]{scrbook}%

%\usepackage[paperwidth=18.89cm,paperheight=24.58cm,twoside,bindingoffset=9mm,outer=2.2cm,inner=1cm,top=2.6cm,bottom=4.5cm]{geometry}

\let\upDelta\Delta
% ********************************************************************
% KOMA-Script setup (todo)
% ********************************************************************
\KOMAoptions{%
%    DIV=15,%
    BCOR=12mm,%
    %paper=b5,%
    fontsize=11pt,%
    cleardoublepage=empty,%
    headsepline=true,
    footinclude=true,%
    headinclude=true,%
    headlines=2.5,%
    open=right,%
    numbers=noenddot%
%    abstract=false%
}

\setlength{\paperwidth}{16.828cm} % set size for latex
\setlength{\paperheight}{26cm}
\special{papersize=16.828cm,26cm} % set size for ghostscript
\typearea[6mm]{1} % 6mm for spine

% ********************************************************************
% Development Stuff
% ********************************************************************
%\listfiles
%\usepackage[l2tabu, orthodox, abort]{nag}
%\usepackage[warning, all]{onlyamsmath}
% ********************************************************************
% Re-usable information
% ********************************************************************

% added by JCC to use Pygmentize 
% http://tex.stackexchange.com/questions/23458/how-to-install-syntax-highlight-package-minted-on-windows-7
% \newcommand\TestAppExists[3]{#2}
\usepackage{minted}


\newcommand{\myTitle}{INTELIGENCIA COMPUTACIONAL Y JUEGOS APLICADOS A LA ENSE\~NANZA\xspace}
\newcommand{\myDegree}{Doctor en Inform\'atica\xspace}
\newcommand{\myName}{Jos\'e Carpio Ca\~nada\xspace}
\newcommand{\myDegreeManualName}{Tesis Doctoral\xspace}
\newcommand{\myProf}{myProf\xspace}
\newcommand{\myOtherProf}{myOtherProf\xspace}
\newcommand{\myDirectorOne}{Juan Juli\'an Merelo Guerv\'os}
\newcommand{\myDirectorTwo}{V\'ictor Manuel Rivas Santos}
\newcommand{\myDirectorThree}{Alberto Prieto Espinosa}
\newcommand{\mySupervisor}{Prof. Dr. D. \myDirectorOne\\Prof. Dr. D. \myDirectorTwo}
\newcommand{\myFaculty}{Escuela T\'ecnica Superior de Inform\'atica y Telecomunicaciones\xspace}
\newcommand{\myDepartment}{Arquitectura y Tecnolog\'ia de Computadores\xspace}
\newcommand{\myDepartmentTwo}{Inform\'atica\xspace}
\newcommand{\myUni}{\protect{Universidad de Granada}\xspace}
\newcommand{\myUniTwo}{\protect{Universidad de Ja\'en}\xspace}
\newcommand{\myLocation}{Granada\xspace}
\newcommand{\myDay}{11}
\newcommand{\myTime}{Noviembre de 2014\xspace}

%*******************************************************
% Packages with options that might require adjustments
%*******************************************************
\usepackage[latin1]{inputenc}
\usepackage[spanish,american,english,es-nodecimaldot]{babel}
\usepackage[square,numbers,sort&compress]{natbib}
\usepackage[spanish]{babelbib}
\usepackage[T1]{fontenc}
\usepackage{textcomp}
\usepackage[mathcal]{euscript}
\usepackage{latexsym}
\usepackage{relsize}
%\usepackage{amsmath} %Incompatible con \usepackage{relsize}
\usepackage{amsfonts}
\usepackage{amssymb}
%\usepackage[sort]{cite}
\usepackage[right]{eurosym}
\usepackage[top]{mcaption}
\usepackage[margincaption]{sidecap}

\usepackage{colortbl}
\usepackage{multirow}
\usepackage{rotating}

\usepackage{lettrine}
\usepackage{textcase}

\usepackage{url}
\usepackage{caption}
\usepackage[tight,spanish]{minitoc}

\usepackage{calc}
\usepackage{fp}

\usepackage{pgfplots}
\usetikzlibrary{dateplot}

\usepackage{cmap}

\usepackage{ifthen}

\usepackage{fancyvrb} %FERGU
\usepackage{etex}
\usepackage{minted}
\usepackage{algorithmic}
\providecommand{\e}[1]{\ensuremath{\times 10^{#1}}}

%\usepackage[a4,frame,center]{crop} %,cross

\input{coloricos} %OJO CON ESTO (EVOSOFT) vs highlight.sty de SOA-SOCO
\usepackage[final]{listings}


% JCC
\usepackage{color}
\pagecolor{white}
\usepackage[final]{pdfpages}

% 
\usepackage[colorinlistoftodos,prependcaption,textsize=tiny]{todonotes}
%\newcommandx{\unsure}[2][1=]{\todo[linecolor=red,backgroundcolor=red!25,bordercolor=red,#1]{#2}}
%\newcommandx{\change}[2][1=]{\todo[linecolor=blue,backgroundcolor=blue!25,bordercolor=blue,#1]{#2}}
%\newcommandx{\info}[2][1=]{\todo[linecolor=OliveGreen,backgroundcolor=OliveGreen!25,bordercolor=OliveGreen,#1]{#2}}
%\newcommandx{\improvement}[2][1=]{\todo[linecolor=Plum,backgroundcolor=Plum!25,bordercolor=Plum,#1]{#2}}
%\newcommandx{\thiswillnotshow}[2][1=]{\todo[disable,#1]{#2}}
%


%\lstset{
%basicstyle=\ttfamily \scriptsize,
%backgroundcolor=\color{ultralightOrange},
%language=c++,
%frame=single,
%stringstyle=\ttfamily,
%showstringspaces=false
%} %DENTRO DE ESO HAY OTRO LISTINGS QUE SE COME LO DE ARRIBA, PERO NO SE SI AFECTA OTRAS COSAS...
%*******************************************************
\usepackage{classicthesis-ldpkg}
%*******************************************************
% Options for classicthesis.sty:
% tocaligned eulerchapternumbers drafting linedheaders listsseparated
% subfigure nochapters beramono eulermath parts minionpro pdfspacing
\usepackage[pdfspacing,eulerchapternumbers,crownquartopaper,%drafting,%linedheaders,%tocaligned,%
            subfigure,beramono,eulermath,parts]{classicthesis2}
% ********************************************************************
% Language/strings for backrefs (change here, thanks, Lorenzo)
%*******************************************************
\renewcommand{\backrefnotcitedstring}{\relax}%(Not cited.)
% Elimina todas estas cosas que est�n comentadas, no hacen m�s que
% estorbar - JJ 
%\renewcommand{\backrefcitedsinglestring}[1]{(Citado en la p\'agina~#1.)}
%\renewcommand{\backrefcitedmultistring}[1]{(Citado en las p\'aginas~#1.)}
%\renewcommand{\backreftwosep}{ y~}
%\renewcommand{\backreflastsep}{ y~}
% ********************************************************************
% Setup and Finetuning
%*******************************************************
\newlength{\abcd} % for ab..z string length calculation
\newcommand{\myfloatalign}{\centering} % how all the floats will be aligned
\setlength{\extrarowheight}{3pt} % increase table row height
% ********************************************************************
% Captions look and feel
%*******************************************************
\DeclareCaptionFont{titlecolor}{\color{titlecolor}\fontfamily{pbk}\selectfont}
\DeclareCaptionFont{alternateTitlecolor}{\color{alternateTitlecolor}\fontfamily{pbk}\selectfont}
\captionsetup{justification=RaggedRight,format=plain,labelsep=newline,font=scriptsize,labelfont=titlecolor,textfont=alternateTitlecolor}%
% ********************************************************************
% Where to look for graphics
%*******************************************************
%\graphicspath{{gfx/}{misc/}} % considered harmful according to l2tabu
% ********************************************************************
% Hyperreferences
%*******************************************************
\hypersetup{%
    colorlinks=true, linktocpage=true, pdfstartpage=3, pdfstartview=FitV,%
    breaklinks=true, pdfpagemode=UseNone, pageanchor=true, pdfpagemode=UseOutlines,%
    plainpages=false, bookmarksnumbered, bookmarksopen=true, bookmarksopenlevel=1,%
    hypertexnames=true, pdfhighlight=/O,%hyperfootnotes=true,%nesting=true,%frenchlinks,%
    urlcolor=colorCorporativoOscuro, linkcolor=alternateTitlecolor, citecolor=Maroon, %pagecolor=RoyalBlue,%
    % uncomment the following line if you want to have black links (e.g., for printing)
    %urlcolor=Black, linkcolor=Black, citecolor=Black, %pagecolor=Black,%
    pdftitle={\myTitle},%
    pdfauthor={\textcopyright\ \myName, \myUni, \myFaculty},%
    pdfsubject={},%
    pdfkeywords={},%
    pdfcreator={pdfLaTeX},%
    pdfproducer={LaTeX with hyperref and classicthesis}%
}
%********************************************************************
% Hyphenation
%*******************************************************
%\hyphenation{Mul-ti-pli-ca-ti-ve de-sa-rro-llar he-te-ro-g�-ne-os pro-pues-ta o-fi-ci-na vi-de-o-ga-mes zoo-kee-per sub-po-pu-la-tion pa-ra-digm Sprin-ger Mi-guel Mo-ra ha-bi-ta-ci�n I-lle-ras Ge-ne-bot pre-sents Ba-ye-sian al-go-rithms E-C-F re-pla-cer re-pla-cer-Na-me He-Ha OS-Gi OS-Gi-Liath W-S-D-L pro-vi-der M-M-D-P}


%********************************************************************
% Redefinir en espa�ol la edici�n en la bibliograf�a
%*******************************************************

\declarebtxcommands{spanish}{%
\def\btxnumeralshort#1{%
${#1}^a$}%\btxnumeralfallback{spanish}{#1}}%
\def\btxnumerallong#1{%
${#1}^a$}%\btxnumeralfallback{spanish}{#1}}%
}

%********************************************************************
% Renombrar al espa�ol algunos identificadores
%*******************************************************

%\renewcommand{\contentsname}{Lista de Contenidos}
%\renewcommand{\listfigurename}{Lista de Figuras}
%\renewcommand{\listtablename}{Lista de Tablas}
%\renewcommand{\bibname}{Bibliograf\'ia}
%\renewcommand{\partname}{Parte}
%\renewcommand{\figurename}{Figura}
%\renewcommand{\tablename}{Tabla}
%\newcommand{\tablaname}{Tabla}
%\renewcommand{\cfttabpresnum}{\tablaname~}

\renewcommand{\person}[1]{{\color{Maroon}{#1}}}

\selectbiblanguage{spanish}

%\renewcommand{\spacedallcaps}[1]{\textsc{\MakeTextUppercase{#1}}}
%\renewcommand{\spacedlowsmallcaps}[1]{\textsc{\MakeTextUppercase{#1}}}

%********************************************************************
% Tipos de letras
%*******************************************************

%\renewcommand{\encodingdefault}{T1}
%\renewcommand{\familydefault}{gm}%papyrus}%bradley} %pbk,ppl
%\renewcommand{\seriesdefault}{b}

%\renewcommand{\rmdefault}{bradley}

\newcommand{\Copyright}{{\small$^\mathrm{\copyright}$}~}
\newcommand{\apriori}{\emph{a priori}\xspace}
\newcommand{\aposteriori}{\emph{a posteriori}\xspace}
\newcommand{\etal}{\emph{et al.}\xspace}
\newcommand{\cronbach}{\mbox{$\alpha$--\emph{Cronbach}}\xspace} % �Esto te sirve para algo? - JJ 

\newcommand{\itemlabel}[1]{{\color{Maroon}{\textsc{#1}}}}

\newcommand{\mysize}{\tiny}

%********************************************************************
% Colores
%*******************************************************

%\definecolor[named]{ultralightOrange}{rgb}{1,.94,.72}
%\definecolor[named]{highlightOrange}{rgb}{1,.86,.29}
%\definecolor[named]{lightOrange}{rgb}{1,.45,0}
\definecolor[named]{darkOrange}{rgb}{.8,.3,0}

\definecolor[named]{minimumOrange}{rgb}{1,.976,.474}
\definecolor[named]{ugrOrange}{rgb}{.945,.365,.165} % PANTONE 179: {R:241, G:93, B:42}
%\definecolor[named]{ugrOrange}{cmyk}{0,.309,.368,0} % PANTONE 179: {C:0, M:79, Y:94, K:0}
\definecolor[named]{ugrGray}{rgb}{.463,.482,.494}

%\definecolorseries{ugrOrangeSerie}{rgb}{last}{ugrOrange}{white}
%\resetcolorseries[10]{ugrOrangeSerie}

\colorlet{ultralightOrange}{ugrOrange!5!minimumOrange}
\colorlet{highlightOrange}{ugrOrange!25!minimumOrange}
\colorlet{lighterOrange}{ugrOrange!40!minimumOrange}
\colorlet{lightOrange}{ugrOrange!75!minimumOrange}

\definecolor{colorCorporativoMasSuave}{named}{ultralightOrange}
\definecolor{colorCorporativoSuave}{named}{highlightOrange}
\definecolor{colorCorporativoMedioSuave}{named}{lighterOrange}
\definecolor{colorCorporativoMedio}{named}{lightOrange}
\definecolor{colorCorporativo}{named}{ugrOrange}
\definecolor{colorCorporativoOscuro}{named}{darkOrange}
\definecolor{Maroon}{named}{colorCorporativoOscuro}

\definecolor{headColor}{named}{white}%{colorCorporativoMasSuave}
\definecolor{titleNumbercolor}{named}{ugrGray}
\definecolor{titlecolor}{named}{ugrOrange}
\definecolor{alternateTitlecolor}{named}{darkOrange}

\newcolumntype{A}{%
>{\color{black}\columncolor{colorCorporativoSuave}}%
p{0.95\textwidth}}

\newcolumntype{B}{%
>{\color{black}\columncolor{colorCorporativoMasSuave}}%
p{0.95\textwidth}}

\newcolumntype{C}{%
>{\columncolor{colorCorporativoSuave}}p{\textwidth}}
%>{\columncolor{colorCorporativo}}p{4pt}}

\newcolumntype{D}{%
>{\color{black}\fontfamily{pbk}\selectfont\columncolor{colorCorporativoSuave}}p{8em}}
%>{\columncolor{colorCorporativo}}p{4pt}

%********************************************************************
% Cabeceras
%*******************************************************

% Mini tablas de contenidos. Paquete minitoc
%\def\ptctitle{Index}
%\def\mtctitle{Index}
%\def\stctitle{Index}
\def\ptctitle{\'Indice}
\def\mtctitle{\'Indice}
\def\stctitle{\'Indice}
\setlength{\mtcindent}{0pt}
\renewcommand{\mtifont}{\normalsize\scshape\lsstyle}

%\typearea[current]{last}
%\typearea[current]{current}
%\setlength{\topmargin}{-4em}
%\setlength{\textheight}{590.80026pt}%{595.80026pt}
%\setlength{\tabcolsep}{1em}




\newboolean{TesisCompleta}
\setboolean{TesisCompleta}{false}

\newboolean{PaginasDeInicio}
\setboolean{PaginasDeInicio}{true}

\newcounter{IncluyeCapitulo}
\setcounter{IncluyeCapitulo}{2}

% \includeonly{Chapters/01-introduccion}

% ********************************************************************
% ********************************************************************
% ********************************************************************
% Comienza el documento
% ********************************************************************
% ********************************************************************
% ********************************************************************
\begin{document}

\frenchspacing
\raggedbottom
%\selectlanguage{spanish}
%\renewcommand{\contentsname}{Lista de Contenidos}
%\renewcommand{\listfigurename}{Lista de Figuras}
%\renewcommand{\listtablename}{Lista de Tablas}
%\renewcommand{\bibname}{Bibliograf\'ia}
%\renewcommand{\partname}{Parte}
%\renewcommand{\figurename}{Figura}
%\renewcommand{\tablename}{Tabla}
%\renewcommand{\figureautorefname}{Figura}
%\renewcommand{\tableautorefname}{Tabla}
%\renewcommand{\paragraphautorefname}{P\'arrafo}
%\renewcommand{\subparagraphautorefname}{Subp\'arrafo}
%\renewcommand{\footnoteautorefname}{Nota al pie}
%\renewcommand{\FancyVerbLineautorefname}{}
%\renewcommand{\theoremautorefname}{Teorema}
%\renewcommand{\appendixautorefname}{Ap\'endice}
%\renewcommand{\equationautorefname}{Ecuaci\'on}
%\renewcommand{\itemautorefname}{Item}
%\newcommand*{\subfigureautorefname}{Figura}

% Par�metros para Lettrine (lo de la primera letra m�s grande en los comienzos de los cap�tulos)
%
\setcounter{DefaultLines}{2}
\renewcommand{\DefaultLoversize}{0.1} %0.25
\renewcommand{\DefaultLraise}{0.3} %0.25
\renewcommand{\DefaultLhang}{0.5} %0.5
\setlength{\DefaultFindent}{0pt}%0.5em}
\setlength{\DefaultNindent}{0pt}%0.1em}
\setlength{\DefaultSlope}{0pt}
\renewcommand{\LettrineFontHook}{\color{colorCorporativoOscuro}\fontfamily{fau}}%{fau}pzc}\fontseries{bx}\fontshape{it}}
\renewcommand{\LettrineTextFont}{\color{colorCorporativoOscuro}\fontfamily{fau}\scshape}
\renewcommand\listfigurename{Lista de Figuras}
\renewcommand\listtablename{Lista de Tablas}
\renewcommand\contentsname{Tabla de Contenidos}
\renewcommand\appendixname{Ap\'endice}
\renewcommand\bibname{Referencias}
\renewcommand{\partname}{Parte}
\renewcommand{\indexname}{Indice}
\renewcommand{\figurename}{Figura}
\renewcommand{\tablename}{Tabla}
\renewcommand{\figureautorefname}{Figura}
\renewcommand{\tableautorefname}{Tabla}
\renewcommand{\paragraphautorefname}{P\'arrafo}
\renewcommand{\subparagraphautorefname}{Subp\'arrafo}
\renewcommand{\footnoteautorefname}{Nota al pie}
%\renewcommand{\FancyVerbLineautorefname}{}
\renewcommand{\theoremautorefname}{Teorema}
\renewcommand{\appendixautorefname}{Ap\'endice}
\renewcommand{\equationautorefname}{Ecuaci\'on}
%\renewcommand{\itemautorefname}{Item}
% \newcommand*{\subfigureautorefname}{Figura}

\pagenumbering{roman}
\pagestyle{scrheadings}%{plain}

\renewcommand{\labelenumii}{\arabic{enumii}.}

% JCC fuerza un salto de linea dentro de la celda de una tabla
\newcommand{\specialcell}[2][c]{%
  \begin{tabular}[#1]{@{}c@{}}#2\end{tabular}}

  
  

%********************************************************************
% Definiciones Captions
%*******************************************************

\sidecaptionvpos{figure}{t}

\ifthenelse{\boolean{PaginasDeInicio}}{
%********************************************************************
% Frontmatter
%*******************************************************



% \include{FrontBackmatter/DirtyTitlepage}
\pagecolor{white}
\pdfbookmark[1]{Portada}{Portada}
%*******************************************************
% Signed Titlepage 
%*******************************************************
\begin{titlepage}
    \begin{center}
        \large
        \vspace*{1.5cm}
        \includegraphics[width=8cm]{gfx/ugr_formal} \\

        \vspace{1.5cm}

		\myDegreeManualName \\ \bigskip
		
        {\color{ugrOrange}\spacedallcaps{\myTitle}} \\ \bigskip
	{\textcolor{ugrGray} {\small Presentado por}} \\ \bigskip
        \spacedlowsmallcaps{\myName}

        \vspace{1.5cm}
 \textcolor{ugrGray}
        {\small Para la obtenci�n del t�tulo de } \normalsize\\
        \large\spacedlowsmallcaps{\myDegree} \\
 \textcolor{ugrGray}		
        {\small en el programa de doctorado } \normalsize\\		
        \large\spacedlowsmallcaps{\myDegreeProgram} \\
\vspace{1.5cm}
\textcolor{ugrGray}{\small Directores }\normalsize\\
        \large\spacedlowsmallcaps{\myDirectorOne}\\
        \large\spacedlowsmallcaps{\myDirectorTwo}\\
        \vspace{2cm}

        {\small Firmado: \myName }\\ \bigskip
	Noviembre 2015


    \end{center}
\end{titlepage}   
%\include{FrontBackmatter/Titlepage}
\include{FrontBackmatter/Titleback}
\cleardoublepage
%%*******************************************************
% Signed Titlepage 
%*******************************************************
\begin{titlepage}
    \begin{center}
        \large
        \vspace*{1.5cm}
        \includegraphics[width=8cm]{gfx/ugr_formal} \\

        \vspace{1.5cm}

		\myDegreeManualName \\ \bigskip
		
        {\color{ugrOrange}\spacedallcaps{\myTitle}} \\ \bigskip
	{\textcolor{ugrGray} {\small Presentado por}} \\ \bigskip
        \spacedlowsmallcaps{\myName}

        \vspace{1.5cm}
 \textcolor{ugrGray}
        {\small Para la obtenci�n del t�tulo de } \normalsize\\
        \large\spacedlowsmallcaps{\myDegree} \\
 \textcolor{ugrGray}		
        {\small en el programa de doctorado } \normalsize\\		
        \large\spacedlowsmallcaps{\myDegreeProgram} \\
\vspace{1.5cm}
\textcolor{ugrGray}{\small Directores }\normalsize\\
        \large\spacedlowsmallcaps{\myDirectorOne}\\
        \large\spacedlowsmallcaps{\myDirectorTwo}\\
        \vspace{2cm}

        {\small Firmado: \myName }\\ \bigskip
	Noviembre 2015


    \end{center}
\end{titlepage}   

%\cleardoublepage
%*******************************************************
% Declaration
%*******************************************************
\refstepcounter{dummy}
\pdfbookmark[0]{Declaration}{declaration}
\chapter*{Visto Bueno}
\thispagestyle{empty}
\bigskip

%\noindent\textit{\myLocation, \myTime}

\vfil

El \textbf{Prof. Dr. D. \myDirectorOne}, Catedr�tico de Universidad del Departamento de \myDepartment de la \myUni y el profesor \textbf{Prof. Dr. D. \myDirectorTwo}, Titular de Universidad del Departamento de \myDepartmentTwo de la \myUniTwo,

\bigskip

\textsc{certifican:}

\bigskip

\noindent Que la memoria titulada:
\begin{center}
``\textbf{\emph{\myTitle}}''
\end{center}
ha sido realizada por \textbf{D. \myName} bajo nuestra direcci�n en el Departamento de \myDepartment de la \myUni para optar al grado de \textbf{\myDegree}.

\vfil

\begin{center}
En \myLocation, a \myDay\xspace de \myTime.
\end{center}
\vfil
Los Directores de la tesis doctoral:\\
\vspace{2cm}

\begin{center}
Fdo. \myDirectorOne \\ y \myDirectorTwo \\
\end{center}

% \begin{flushright}
%     \begin{tabular}{m{5cm}}
%         \\ \hline
%         \centering\myName \\
%     \end{tabular}
% \end{flushright}

\cleardoublepage
\include{FrontBackmatter/Declaration}
\cleardoublepage
%*******************************************************
% Dedication
%*******************************************************
\thispagestyle{empty}
%\phantomsection
\refstepcounter{dummy}
\pdfbookmark[1]{Dedicatoria}{Dedicatoria}

\vspace*{3cm}

\begin{flushright}
A mi familia: \hspace{3em}\\
Manuel Angel, Matilde, Manuel Jos� y Fernando.\\
\end{flushright}



\medskip

\cleardoublepage
%*******************************************************
% Abstract
%*******************************************************
%\renewcommand{\abstractname}{Abstract}
% \pdfbookmark[1]{Abstract}{Abstract}
\begingroup
\let\clearpage\relax
\let\cleardoublepage\relax
\let\cleardoublepage\relax



\pdfbookmark[1]{Resumen}{Resumen}
% \chapter*{Abstract}


\chapter*{Resumen}

% \section{Resumen}

Los m�todos tradicionales de ensa�anza pueden influenciar de forma negativa en la motivaci�n y en las expectativas de los estudiantes,
debido entre otros a la comunicaci�n unidireccional, las metodolog�as
r�gidas o los enfoques orientados a resultados, provocando una reducci�n del rendimiento acad�mico. Con el objetivo de hacer que el proceso 
de aprendizaje sea motivante esta tesis presenta una metodolog�a que
permite mejorar la experiencia de aprendizaje de alumnos y profesores. Como aplicaci�n pr�ctica de la metodolog�a propuesta, se han llevado a cabo varias experiencias
reales en asignaturas clasicas de Ingeligencia Artificial. En estas asignaturas se han sustitu�do algunas sesiones por la participaci�n
en competiciones nacionales e internacionales de juegos basados en Ingeligencia Artificial que ten�an como objetivo la realizaci�n de un
 agente capaz de competir contra otros adversarios. Se han analizado entre otros elementos el ranking en la competici�n,
la opini�n de los estudiantes o el progreso acad�mico con el fin de evaluar la metodolog�a empleada. Hemos comprobado como la experiencia
educacional mojora la percepci�n global de los estudiantes, mejorando incluso sus resultados acad�micos y sus habilidades personales.
 Como conclusi�n, esta metodolog�a nos ha permitido comprobar que el proceso es m�s importante que el resultado y que es posible adaptarla
 a diferentes escenarios de aprendizaje dentro de una instituci�n acad�mica.
 ~\\


\vfill


\endgroup

\vfill
\cleardoublepage
% JCC
% %*******************************************************
% Publications
%*******************************************************
\pdfbookmark[1]{Publicaciones}{Publicaciones Cient�ficas}
\chapter*{Scientific Publications}
Algunas de las ideas, im�genes y datos mostrados en esta t�sis han sido publicados previamente en las siguientes referencias:
\bigskip

\begin{itemize}
 \item \person{Pablo Garc�a-S�nchez, J. Gonz�lez, Pedro A. Castillo, Maribel Garc�a Arenas, Juan Juli�n Merelo Guerv�s} \emph{Service oriented evolutionary algorithms}.  Soft Comput. 17(6): 1059--1075 (2013).


% no metas estos del "companion", que todos sabemos lo que es... 

~\\

 \item \person{V\'ictor M. Rivas, J.J. Merelo, I. Rojas, G. Romero, P.A. Castillo,  J. Carpio} \emph{ Evolving two-dimensional fuzzy systems}, \textbf{Fuzzy Sets and Systems}, 2003; Factor de impacto: \textbf{0,577};\\

 \item \person{Jos\'e Carpio, Pablo Garc\'ia-S\'anchez, Antonio Miguel Mora, Juan Juli\'an Merelo Guerv\'os, Jes\'us Caraballo, Ferm\'in Vaz, Carlos Cotta} \emph{ Evolving the Strategies of Agents for the ANTS Game}. International Work-Conference on Artificial Neural Networks (IWANN)}, 2013
 
 \item \person{Pablo Garc�a-S�nchez, J. Gonz�lez, Pedro A. Castillo, Juan Juli�n Merelo Guerv�s, Antonio Miguel Mora, Juan Lu�s Jim�nez Laredo, Maribel Garc�a Arenas} \emph{ A Distributed Service Oriented Framework for Metaheuristics Using a Public Standard}. In Proceeding of Nature Inspired Cooperative Strategies for Optimization. Studies in Computational Intelligence. Springer, 2010. p: 211--222.


 \item \person{P. Garc�a-S�nchez, A. Fern�ndez-Ares, A. M. Mora, P. A. Castillo, J. Gonz�lez and J.J. Merelo} \emph{Tree depth influence in Genetic Programming for generation of competitive agents for RTS games}. Applications of Evolutionary Computation, EvoApplicatons 2010: EvoCOMPLEX, EvoGAMES, EvoIASP, EvoINTELLIGENCE, EvoNUM, and EvoSTOC, Proceedings. Springer, 2014. Lecture Notes in Computer Science (to appear).

 \item \person{P. Garc�a-S�nchez, A. M. Mora, P. A. Castillo, J. Gonz�lez and J.J. Merelo} \emph{A methodology to develop Service Oriented Evolutionary Algorithms}. Proceedings of 8th International Symposium on Intelligent Distributed Computing (IDC'2014) Springer, 2014. Studies in Computational Science (to appear).

\end{itemize}

\cleardoublepage
%*******************************************************
% Acknowledgments
%*******************************************************
\pdfbookmark[1]{Agradecimientos}{agradecimientos}

\begin{flushright}
 
\end{flushright}



\bigskip

\begingroup
\let\clearpage\relax
\let\cleardoublepage\relax
\let\cleardoublepage\relax
\chapter*{Agradecimientos}


\endgroup




\pagestyle{scrheadings}
\cleardoublepage
%*******************************************************
% Table of Contents
%*******************************************************
%\phantomsection
\refstepcounter{dummy}
% \pdfbookmark[1]{\contentsname}{tableofcontents}
\pdfbookmark[1]{Tabla de contenidos}{Tabla de contenidos}
\setcounter{tocdepth}{2}
\dominitoc
\tableofcontents
%\markboth{\spacedlowsmallcaps{\contentsname}}{\spacedlowsmallcaps{\contentsname}}
\markboth{\textsc{\contentsname}}{\textsc{\contentsname}}
%*******************************************************
% work-around to have small caps also here in the headline
% will not work at this place if the TOC has more than 2 pages
% use \manualmark and then the \markboth as above
% later a modification of \automark[section]{chapter}
%*******************************************************
% List of Figures and of the Tables
%*******************************************************
\clearpage

\begingroup
    \let\clearpage\relax
    \let\cleardoublepage\relax
    \let\cleardoublepage\relax
    %*******************************************************
    % List of Figures
    %*******************************************************
    %\phantomsection
    \refstepcounter{dummy}
    %\addcontentsline{toc}{chapter}{\listfigurename}
    \pdfbookmark[1]{\listfigurename}{lof}
	% \pdfbookmark[1]{Lista de figuras}{Lista de figuras}
    \listoffigures

    \vspace*{8ex}

    %*******************************************************
    % List of Tables
    %*******************************************************
    %\phantomsection
    \refstepcounter{dummy}
    %\addcontentsline{toc}{chapter}{\listtablename}
    \pdfbookmark[1]{\listtablename}{lot}
	% \pdfbookmark[1]{Lista de tablas}{Lista de tablas}
    \listoftables

    \vspace*{8ex}
%   \newpage

    %*******************************************************
    % List of Listings
    %*******************************************************
%   %\phantomsection
%   \refstepcounter{dummy}
%   %\addcontentsline{toc}{chapter}{\lstlistlistingname}
%   \pdfbookmark[1]{\lstlistlistingname}{lol}
%   \lstlistoflistings

    %*******************************************************
    % Acronyms
    %*******************************************************
    %\phantomsection
    \refstepcounter{dummy}
    \pdfbookmark[1]{Acr\'onimos}{Acr\'onimos}
    \chapter*{Acr\'onimos}
    \begin{acronym}[CSCW]
	\acro{AAAI}{Association for the Advancement of Artificial Intelligence}
    \acro{AI}{Artificial Intelligence}	
	\acro{CIAR}{Competici\'on de Inteligencia Artificial y Rob\'otica}
	\acro{CITIC}{Centro de Investigaci\'on en Tecnolog\'ias de la Informaci\'on y de las Comunicaciones}
	\acro{DIESIA}{Departamento de Ingenier\'ia Electr\'onica Sistemas Inform\'aticos y Atom\'atica}
	\acro{DSCL}{Desaf\'io Solar Costa de la Luz}
	\acro{DTI}{Departamento de Tecnolog\'ias de la Informaci\'on}
	\acro{EPFL}{\'Ecole polytechnique f\'ed\'erale de Lausanne}
	\acro{FLL}{First Lego League}
	\acro{GA}{Genetic Algorithm}
	\acro{HWO}{Hello World Open}
	\acro{IA}{Inteligencia Artificial}
	\acro{IC}{Inteligencia Computacional}
	\acro{IAeIC}{Inteligencia Artificial e Ingenier\'ia del Conocimiento}
	\acro{IIA}{Introducci\'on a la Inteligencia Artificial}
	\acro{JCR}{Journal Citation Index}
	\acro{LIA}{Laboratorio de Inteligencia Artificial}
	\acro{MAC}{Modelos Avanzados de Computaci�n}
	\acro{MD}{Miner\'ia de Datos} 
	\acro{MIT}{Massachusetts Institute of Technology}
    \acro{MLExAI}{Machine Learning Experience in AI}
	\acro{PD}{Programaci\'on Declarativa}
	\acro{PCB}{Printed Circuit Board}
	\acro{RC}{Representaci\'on del Conocimiento}
	\acro{SOM}{Self Organizing Maps}
	\acro{VARK}{Visual Auditivo Lectura/Escritura Quinest\'atico)}
	
    \end{acronym}
\endgroup

\cleardoublepage 
}
{

}


\pdfbookmark[0]{Cap\'itulos}{Cap\'itulos}
% *******************************************************
% Mainmatter
% *******************************************************
\pagenumbering{arabic}

\ifthenelse{\boolean{TesisCompleta}}{
%-----------------------------
\myPart{Cap\'itulos}\label{part:capitulos}

%REALMENTE tienes que escribir una introducción. 
% Aqu� quedan todav�a cosas rarunas - JJ
}
{
%-----------------------------  PONER LETTRINE EN SECCIONES Y NO INDENT EN SUBSECCIONES!!!
\myPart{Cap\'itulos}\label{part:capitulos}

%----- Modificado por JCC
% -- \ifthenelse{\equal{\value{IncluyeCapitulo}}{2}}{\include{Chapters/00-intro}}{\myChapter{Introduction}\label{chap:introduction}}
\ifthenelse{\equal{\value{IncluyeCapitulo}}{2}}{\myChapter{Introducci{\'o}n}\label{chap:introduccion}

\begin{flushright}{\slshape
"Un hombre va al saber como a la guerra: bien despierto, con miedo, con respeto y con absoluta confianza. Ir
en cualquier otra forma al saber o a la guerra es un error, y quien lo cometa vivir� para lamentar sus pasos."
 } \\ \medskip
    --- {Las ense�anzas de don Juan, Una forma Yaqui de conocimiento (Domingo, 20 de agosto, 1961); Carlos Castaneda}
\end{flushright}
\minitoc\mtcskip
\vfill

\section{Introducci�n}

\lettrine{L}{a} presente tesis doctoral introduce una metodolog�a para mejorar la experiencia tanto del docente como del
 estudiante en el aprendizaje de la Inteligencia Computacional (IC) a trav�s de la participaci�n en juegos competitivos. 
 La Inteligencia Computacional es una rama de la Artificial (IA) que combina elementos de aprendizaje, adaptaci�n, evoluci�n y l�gica difusa para crear programas inteligentes. 
 La investigaci�n en Inteligencia Computacional no se apoya tanto en la estad�stica y utiliza t�cnicas como las Redes Neuronales, la Computaci�n Evolutiva,
 la Inteligencia Emergente o Swarm Intelligence, los Sistemas Inmunes Artificiales o los Sistemas Difusos. 
 
Los m�todos did�cticos tradicionales a menudo conllevan varios inconvenientes debido a las limitaciones de la 
ense�anza formal \cite{Novosadova2007}. Entre otros, la comunicaci�n unidireccional deja de fomentar la participaci�n
 activa de los estudiantes, de esta forma los profesores deben realizar un esfuerzo adicional en el camino de alcanzar los 
 objetivos did�cticos propuestos. Los errores cometidos por el alumno en el proceso de aprendizaje suelen castigarse desde 
 un enfoque punitivo del m�todo de ense�anza. Adem�s de esto, los calendarios suelen ser r�gidos y no siempre se adaptan a 
 las necesidades de los estudiantes con diferentes niveles de conocimientos y habilidades individuales \cite{dib1988formal}. Todo esto
 deriva en una disminuci�n en la motivaci�n y el inter�s de los estudiantes, que se agrava en la educaci�n en ingenier�a 
 \cite{van2009motivating}. Una metodolog�a m�s flexible es especialmente adecuada en el caso de materias que 
 incluyan cr�ditos pr�cticos. En este sentido, los m�todos interactivos (por ejemplo: las sesiones de resoluci�n de problemas, 
 las pr�cticas con ordenador y los juegos) permiten a los profesores conseguir una mayor implicaci�n de los estudiantes en las 
 actividades propuestas \cite{adams2000instructional}. Teniendo todo esto en cuenta, el aprendizaje mediante el juego llega
 a la escena de la ense�anza como una de las experiencias de aprendizaje m�s exitosas \cite{Veganzones2011}. ~\\
 
La educaci�n formal est� caracterizada por un modelo sistem�tico, estructurado y guiado por una serie de directivas curriculares,
 con frecuencia presentando objetivos, contenidos y metodolog�as r�gidas tanto para los profesores como para los estudiantes 
 \cite{dib1988formal}. Adem�s, el aprendizaje formal no se ajusta a nuestra manera natural de aprender, solo se muestra adecuado para un 18\%
 de los ni�os de primaria y secundaria (K-12\footnote{K-12 es el t�rmino en Estados Unidos y Canad� para los estudiantes de primaria y secundaria,
 desde 4-6 a 17-19 a�os}) y un 5.1\% de los estudiantes universitarios \cite{banks2007and}. Con esta configuraci�n, la educaci�n formal 
 no siempre es un buen est�mulo a medida que el estudiante demanda mayor naturalidad, flexibilidad e interacci�n para apoyar su 
 aprendizaje. Adem�s los estudiantes entran en la escena del aprendizaje con diferentes grados de compromiso, habilidades y estilos
 de did�cticos, hecho que afecta a su grado de motivaci�n \cite{kirkland2008games}.  Como indican las �ltimas teor�as, la
 motivaci�n representa un factor clave para aprender y obtener un resultado acad�mico exitoso \cite{amrai2011relationship, 
 maclellan2005academic, williams2011five}. ~\\
 
En contraposici�n a lo anterior, la educaci�n informal da a los estudiantes la oportunidad de participar en su aprendizaje de forma 
proactiva a trav�s metodolog�as flexibles y con diferentes estilos de aprendizaje \cite{chen2012investigating}. Esto ampl�a las competencias
 personales m�s que las desarrolladas en el aprendizaje formal (por ejemplo, el liderazgo, la disciplina, la responsabilidad, el 
 trabajo en equipo, la gesti�n de conflictos, la planificaci�n, la organizaci�n  o las relaciones interpersonales). Como consecuencia,
 es considerada por los estudiantes una metodolog�a m�s favorable, eficaz y estimulante comparada con una educaci�n formal menos 
 atractiva y eficiente \cite{schulz2008importance}. La educaci�n informal y el juego est�n cambiando el modo en el que pensamos sobre el
 conocimiento y el aprendizaje, adem�s de la forma en la que estructuramos el trabajo y las ideas. El aprendizaje a trav�s del juego
 permite al alumnado construir su propio conocimiento, basado en la comprensi�n de sus propias experiencias, tal y como indican las
 recientes teor�as constructivistas \cite{gagnon2005constructivist}. El aprendizaje activo es eficaz para motivar y mejorar el rendimiento de los
 estudiantes, promoviendo el pensamiento creativo y con diferentes estilos de aprendizaje. El estilo quinest�tico (t�rmino que hace
 referencia al aprendizaje a trav�s de actividades f�sicas) es el m�s adoptado en juegos, pero el estilo VARK (visual, auditivo, 
 lectura/escritura y quinest�tico) tambi�n puede ser utilizado \cite{newble2013handbook}. El aprendizaje a trav�s del juego est� mejor
 documentado para ni�os de primaria y secundaria (K-12) que para universitarios \cite{rice2009playful}. Las ventajas del aprendizaje interactivo
 para adultos son claras y variadas, especialmente en ingenier�a d�nde el conocimiento pr�ctico requiere de una interacci�n directa en
 los laboratorios adem�s de las lecciones te�ricas \cite{rieber2001designing}. ~\\
 
Desde el a�o 2010 se utiliza el concepto de gamificaci�n \cite{llagostera2012gamification} para referirnos al uso de juegos en ambientes o entornos no
 l�dicos \cite{deterding2011game}. Tambi�n podemos encontrar el t�rmino ludificaci�n en el mismo sentido. La gamificaci�n o ludificaci�n en la ense�anza,
 fomenta de forma activa la creatividad, el desarrollo de estrategias para
 la resoluci�n de problemas y la autoconfianza para abordar nuevos desaf�os \cite{lester2008play}. Sin embargo, la experiencia no es
 siempre suficiente para aprender y es necesario incorporar otros aspectos en el proceso \cite{bolton2010reflective}, como pueden ser, la observaci�n,
 el an�lisis, el pensamiento cr�tico, la abstracci�n y los ensayos del conocimiento adquirido en nuevas situaciones. En este contexto,
 el aprendizaje basado en juegos competitivos proporciona un escenario adecuado para proporcionar todos los elementos necesarios para alcanzar
 un aprendizaje constructivo.  Sin embargo, un porcentaje muy bajo de profesores y estudiantes se aprovechan de la gran popularidad de los
 juegos con fines educativos. ~\\
 
El aprendizaje activo a trav�s de los juegos competitivos se ha probado como un factor motivador permitiendo a los estudiantes adquirir conocimiento
 por ellos mismos a trav�s de la actividad y el razonamiento \cite{carpio2011}. Este modo de aprendizaje se caracteriza por una
 perspectiva centrada en el estudiante d�nde el proceso es m�s importante que el resultado. Por tanto, los profesores se convierten en el 
 medio para guiar a los estudiantes en el proceso de aprendizaje, d�nde los alumnos motivados aprenden las materias de la asignatura a trav�s 
 de la resoluci�n de los desaf�os planteados \cite{hmelo2004problem}. Como principal ventaja, los estudiantes responden de manera natural a este
 tipo de aprendizaje, d�nde los juegos ofrecen un medio para formar y reformar ideas de una manera divertida e interactiva. Como resultado,
 cuanto m�s motivado e implicado est� el alumno, mayor es el aprendizaje \cite{squire2003harnessing}.

\section{Aprender jugando en Inteligencia Artificial}
\label{sec:intro:jugando}
Desde que Alan Turing estableciese el primer juego que pod�a ser jugado de forma autom�tica por m�quinas utilizando algoritmos l�gicos, estos 
han sido utilizados como una metodolog�a de aprendizaje para ense�ar diferentes conceptos de IA \cite{turing1950computing}. Esto transform� los juegos en 
una herramienta potencialmente exitosa utilizada para ense�ar una gran variedad de m�todos pr�cticos gracias a su habilidad para motivar a los 
estudiantes proporcionando espontaneidad, flexibilidad e interactividad para apoyar la experiencia de aprendizaje \cite{moursund2006introduction}. Los ejemplos
 m�s representativos en educaci�n son los juegos de mesa cl�sicos como Backgammon, utilizado para ense�ar m�todos de exploraci�n por la t�cnica 
 de aprendizaje por refuerzo \cite{moursund2006brief}; Checkers, utilizado para desarrollar t�cnicas de resoluci�n de problemas basadas en b�squedas
 \cite{sturtevant2008analysis}; Tic-Tac-Toe, utilizado para Mini-Max y poda Alfa-Beta \cite{michulke2011distance}; N-puzle, utilizado en b�squeda en espacio
 de estados \cite{markov2006pedagogical}; o n-Reinas, utilizado para ense�ar problemas de satisfacci�n de restricciones 
 \cite{letavec2002n}, adem�s de otros. ~\\
 
La motivaci�n de los estudiantes juega un papel clave en el aprendizaje y demuestra que es posible alcanzar los objetivos
 acad�micos con �xito a trav�s de los desaf�os propuestos en las asignaturas. Por ejemplo, The Open Racing Car Simulator, un entorno de trabajo de
 c�digo abierto
 y multi\-plataforma altamente portable, ha sido utilizado como un juego de coches en tres dimensiones (3D) para ense�ar principios mec�nicos en la
 Universidad Northern
 Illinois \cite{coller2009video}. Adem�s, en la Universidad Nacional de Maynooth se han organizado diferentes ligas RoboCode con el objetivo de ense�ar
 lenguajes de programaci�n \cite{o2006robocode}. En estos casos, a los alumnos se les plantea el dise�o de agentes inteligentes, llamados
 "robots" o simplemente "bots", para competir unos con otros intentando imitar el comportamiento humano \cite{eisenstein2003evolving}. En otros casos, la competici�n ayuda a
 descubrir estudiantes con talento y habilidades especiales en las escuelas de ingenier�a. Como ejemplo, la competici�n internacional Facebook
 Hacker Cup comenz� en 2011 con este prop�sito, y consisti�  en resolver un n�mero de problemas basados en algoritmos utilizando cualquier 
 framework o lenguaje de programaci�n \cite{forivsek2013pushing}. Adem�s, la universidad del estado de Wichita ha utilizado Lego Mindstorm para la First
 Lego League, una competici�n que ha sido probada como una metodolog�a �til de ense�anza en estudiantes K-12. En este caso, ha permitido adquirir 
 aptitudes individuales, valores, habilidades y conocimientos que han sido incorporados de forma natural gracias a la educaci�n informal 
 \cite{whitman2003using}.~\\
 
Con el objetivo de utilizar motivar la formaci�n en IA y la investigaci�n en este campo, han surgido
 diferentes competiciones tanto nacionales como internacionales. Por ejemplo, la Universidad de Stanford utiliz� la AAAI GGP (Association for the 
 Advancement of AI General Game Playing) como una excelente plataforma de desarrollo para estudiantes en
 verano de 2005 \cite{genesereth2005general}. Adem�s, la Universidad de Hartfold ha desarrollado y probado un conjunto de proyectos denominados 
 MLExAI (Machine Learning Experience in AI) que pueden ser integrados en cursos introductorios para ense�ar aprendizaje
 autom�tico en IA \cite{markov2006pedagogical}. La Universidad de Essex lanz� la liga MS Pac-Man contra Ghost con el fin de enfrentar a robots
 creados por diferentes competidores que hab�an sido previamente probados con �xito por profesores y estudiantes en cursos de IA 
 \cite{szita2007learning}. Otra competici�n que se ha celebrado tradicionalmente en Universidades ha sido el Physical Travelling Salesman Problem,
 un juego con un �nico jugador dirigido a resolver problemas de optimizaci�n combinatoria con controladores de IA \cite{perez2012physical}; 
 La competici�n de Carreras de Coches Simuladas, un evento que consiste en tres competiciones d�nde se aplican t�cnicas de 
 IA para controlar coches en un juego de carreras \cite{loiacono20102009}; la competici�n de IA Mario, ha sido un referente utilizado en 
 diferentes competiciones relacionadas con congresos internacionales en educaci�n y/o investigaci�n \cite{karakovskiy2012mario}; y la 
 competici�n Start Craft AI, un juego avanzado de estrategia para los que los robots con IA tienen que abatir a jugadores humanos expertos 
 en tiempo real \cite{togelius2010multiobjective}, adem�s de otros. ~\\
 
Estos ejemplos presentan un escenario d�nde la observaci�n, la abstracci�n de conceptos, el pensamiento cr�tico, el an�lisis y el 
conocimiento adquirido concurren en un proceso educativo de �xito dentro del contexto de una competici�n. En esta l�nea, la competici�n AI Challenge 
organizada por la Universidad de Waterloo y patrocinada por Google \cite{savchuk2012analysis} se muestra como una herramienta muy �til en la ense�anza
de la IA. Se han celebrado diferentes ediciones de esta competici�n,
y cada una de ellas ha estado centrada en un reto distinto d�nde  los
 participantes concursaban online contra agentes de otros competidores \cite{perick2012comparison}. La primera edici�n (Rock-Paper-Scissors, oto�o 2009) se 
 centr� en un juego muy conocido, sin embargo, las siguientes competiciones (Tron Light-Cycles en la primavera de 2010, Planet Wars en oto�o de 2010 y 
 Ants en oto�o de 2011) se basaron en dise�o de juegos originales. Esto proporcion� a estudiantes de todo el mundo un factor de motivaci�n extra para 
 explorar nuevos enfoques, experimentar con ideas diferentes y 
 finalmente encontrar soluciones a problemas. Google AI Challenge se distingue por ser una competici�n internacional \- con partidas online
 multijugador \- tanto para universitarios como para profesionales. Ha sido utilizado para ense�ar una 
 variedad de algoritmos de IA (p. ej. algoritmos gen�ticos, redes neuronales o l�gica borrosa), a la vez que se ense�ar nuevos lenguajes de 
 programaci�n a trav�s de la implementaci�n de los agentes inteligentes \cite{carpio2014}. ~\\

\section{Competiciones de rob�tica aplicadas a la Inteligencia Artificial}
\label{sec:intro:robotica}
 Las competiciones de rob�tica han demostrado ser una potente herramienta para consiguir que los estudiantes ganen inter�s en la IA y su aplicaci�n
 a los agentes rob�ticos. Las competiciones tambi�n ofrecen a los estudiantes la oportunidad de encontrase con
 estudiantes de otros lugares que aportan soluciones diferentes a los retos planteados. Adem�s, mediante este tipo de experiencias, se acerca a los
 estudiantes a los problemas que se plantean en la vida ral. Puede ser muy diferente dise�ar
 y programar c�digo para una simulaci�n, que hacerlo para un robot real m�vil. Por ejemplo, hay muchos factores adicionales que hay que tener
 en cuenta, como la carga de la bater�a o la iluminaci�n del lugar de la competici�n. Muchas cuestiones surgen durante la experiencia, tanto en
 relaci�n con el dise�o del hardware, como del software. ~\\
 
Los profesores de Ingenier�a tienen como principal objetivo de formar a los ingenieros del ma�ana y, en esa formaci�n, es muy
 importante acercar a los estudiantes al mundo real. Esta idea se ha plasmado en varias experiencias relevantes que se pueden encontrar en 
 la literatura. Por ejemplo, en Zhongli y otros \cite{wang2007internet}, se presenta una plataforma basada en Internet para una competici�n de
 f�tbol dedicada a robots
 educacionales. De Vault \cite{de1998competition}, describe una experiencia a trav�s de un curso de robots m�viles y la participaci�n en una 
 competici�n anual. En Grimes y otros \cite{grimes2008robotics}, se describen los resultados acad�micos y los conocimientos adquiridos por los
 estudiantes en una competici�n que se convierte en una 
 excelente oportunidad de crecimiento educacional. Es este trabajo se describe c�mo los estudiantes tienen toda la responsabilidad en la 
 definici�n de las reglas de la competici�n, dise�ando y construyendo la pista y organizando los deferentes aspectos de la competici�n. En Berlier 
 y otros \cite{berlier2009robot}, se presenta
 una metodolog�a utilizada para reemplazar m�todo de evaluaci�n tradicional por otro d�nde los estudiantes construyen un robot basado en en micro 
 controlador con 
 el objetivo de participar en una competici�n.  Murphy \cite{murphy2001strategy}, describe una estrategia para integrar una competici�n de dise�o 
 de robots en el aula 
 como una forma de mejorar la experiencia de aprendizaje mejorar el desarrollo intelectual. Finalmente en Almeida y otros \cite{almeida2000mobile},
 se presentan las 
 competiciones de robots m�viles como un evento muy adecuado para la experimentaci�n, la investigaci�n y el desarrollo en muchas �reas de la
 educaci�n secundaria y Universitaria.
 
 
\section{Descripci�n de las competiciones}
\label{sec:intro:competiciones}

Durante el desarrollo de esta tesis se ha participado en 6 competiciones Cosmobot 2009, CIAR 2010, Ants 2011, First Lego League 2011/2012 y 
2012/2013 y Hello World Open 2014. De las seis competiciones en las que se ha participado se han descrito en forma de art�culos dos de ellas
 Cosmobot 2009 (Carpio y otros 2011 \cite{carpio2011}) y Ants 2011 (Carpio y otros 2014 \cite{carpio2014} y Carpio y otros 2013 
 \cite{carpio2013}). A continuaci�n describiremos brevemente cada una
 de ellas. ~\\
 
\subsection{Cosmobot 2009}
Cosmobot es la primera de las iniciativas de gamificaci�n en el aula de este trabajo de investigaci�n. Se financia gracias a una ayuda de la
 Universidad de Huelva para fomentar proyectos de innovaci�n educativa. La competici�n que se celebr� en Madrid en la sede de CosmoCaixa los 
 d�as 28 y 29 de marzo de 2009. En esta edici�n se presentaban dos pruebas diferentes: luchadores de sumo y velocistas seguidores de linea. La 
 prueba de velocistas descrita en \cite{carpio2011}, consist�a en recorrer a m�xima velocidad un circuito dibujado con l�neas  negras sobre fondo blanco
 que serv�an de gu�a a los robots participantes. El robot, utilizando alg�n tipo de sensor ten�a que reconocer estas l�neas y seguirlas a la mayor
 velocidad posible sin llegar a salir de la pista, delimitada por l�neas rojas. ~\\

\begin{table}[]
\centering
\caption{Cosmobot 2009}
\label{cosmobot-2009}
\begin{tabular}{|l|} \hline 
�mbito: Nacional  ~\\
Lugar: CosmoCaixa Madrid  ~\\
A�o: 2009  ~\\
Modalidades: Robots seguidores de l�neas y luchadores de sumo  ~\\
N�mero de participantes: 29  ~\\
Web: http://www.roboticspot.com/especial/cosmobot2009/  ~\\
Publicaci�n:  Carpio y otros 2011 \cite{carpio2011}  ~\\ \hline
\end{tabular}
\end{table}

\subsection{Competici�n de Inteligencia Artificial y Rob�tica 2010}
Competici�n de Inteligencia Artificial y Rob�tica 2010 (CIAR 2010), celebrada en la Universidad de Huelva y organizada por profesores de los departamentos 
de Tecnolog�as de la Informaci�n (DTI) y de Ingenier�a Electr�nica, Sistemas Inform�ticos y Autom�tica (DIESIA) de la 
 Universidad de Huelva. En esta competici�n se celebraron tres modalidades: dise�o de robots, seguidores de l�nea y carreras de coches en entorno 
 simulado. La modalidad de dise�o de robots premiaba la creatividad de los dise�os, la originalidad, la funcionalidad y el modo de construcci�n 
 de los prototiopos. La prueba de seguidores de l�nea consist�a, al igual que en Cosmobot en seguir un circuito dibujado con l�neas a la 
 mayor velocidad posible. Y por �ltimo la competici�n de carreras simuladas de coches consisti� en la programaci�n de un agente que controlase 
 un coche de carreras en un entorno virtual 3D. ~\\

\begin{table}[]
\centering
\caption{Competici�n de Inteligencia Artificial y Rob�tica 2010}
\label{ciar-2010}
\begin{tabularx}{\textwidth}{|X|} \hline  
�mbito: Local \\
Lugar: Escuela T�cnica Superior de Ingenier�a, Universidad de Huelva \\
A�o: 2010 \\
Modalidades: Robots seguidores de l�neas, carreras \\
    de coches virtuales, dise�o de robots. \\ 
N�mero de participantes: 12 \\
Web: http://goo.gl/upakE0 \\ \hline
\end{tabularx}
\end{table}

\subsection{First Lego League}
La First Lego League es una competici�n internacional con pruebas de diferentes �mbitos. En cada edici�n se plantea un reto diferente que hay que
 resolver utilizando una serie
 de componentes de la compa��a Lego y su unidad de control Lego Mindstorms. Cada a�o la tem�tica de la competici�n es diferente. En el a�o 2011 la
 competici�n ten�a el t�tulo de "Food Factor�� y estaba relacionado con la problem�tica de la alimentaci�n, producci�n, almacenamiento o distribuci�n
 de alimentos. La edici�n 2012 ten�a como t�tulo "Senior solutions�� y pretend�a motivar a los participantes a reflexionar sobre las necesidades de
 los mayores y a pensar en posibles soluciones. ~\\

\begin{table}[]
\centering
\caption{First Lego League}
\label{fll}
\begin{tabularx}{\textwidth}{|X|} \hline  
 �mbito: Provincial/Internacional \\
Lugar: Universidad de Huelva \\
A�o: 2011/2012 y 2012/2013 \\
Modalidades: Dise�o de robots para resolver un reto  \\
N�mero de participantes: 100 (aproximadamente) \\
Web: http://www.firstlegoleague.es/ \\ \hline
\end{tabularx}
\end{table}

\subsection{Artificial Intelligence Challenge Ants 2011}
Competici�n organizada por la Universidad de Waterloo y con el patrocinio de Google. El a�o 2011 se celebra su cuarta edici�n  siendo las anteriores:
 Rock Paper Scissors \- oto�o 2009, Tron \- invierno 2010, Planet Wars \- oto�o 2010, Ants \- oto�o 2011. El reto de la cuarta edici�n consist�a en
 organizar una comunidad de hormigas con el objetivo de conquistar los hormigueros enemigos. Sobre un mapa se sit�an diferentes comunidades de
 hormigas, cada una de ellas con un n�mero de hormigueros. En el mapa se distribuye comida de forma aleatoria que las hormigas pueden capturar.
 Cada vez que una hormiga alcanza la comida, del hormiguero sale una nueva hormiga. De esta forma se consigue que la comunidad de hormigas crezca. 
 Sin embargo el objetivo �ltimo de la prueba no es que la comunidad sea muy grande, sino que se lleguen a conquistar los hormigueros enemigos. Gana
 el equipo que consigue conquistar m�s hormigueros. En este caso era importante dise�ar estrategias de defensa, de ataque, de captura de alimentos y de
 captura de hormigueros enemigos. Adem�s las estrategias deb�an ser muy r�pidas ya que la ejecuci�n se organizaba en turnos de 1000 ms, lo que requer�a
 un gran esfuerzo de optimizaci�n de los algoritmos. Cabe destacar de esta competici�n que en ella participaban estudiantes de todos los rincones del
 mundo, trabajadores de empresas prestigiosas como Google, y estudiantes de las mejores universidades del mundo Stanford, MIT, EPFL entre otras. ~\\
 
\begin{table}[]
\centering
\caption{IA Challenge Ants 2011}
\label{ants-2011}
\begin{tabularx}{\textwidth}{|X|} \hline  
�mbito: Internacional  ~\\
Lugar: Online, sitio web de la organizaci�n  \\
A�o: 2011  \\
Modalidades: Manejo de una comunidad de agentes rob�ticos  \\ 
N�mero de participantes: 7.897  \\
Web: http://ants.aichallenge.org/  \\
Publicaciones: Carpio y otros 2013 \cite{carpio2013} y Carpio y otros 2014 \cite{carpio2014} \\ \hline
\end{tabularx}
\end{table}

\subsection{Hello World Open Competition 2014}
En esta ocasi�n el reto consisti� en programar un agente rob�tico capaz de correr en una pista virtual tipo Scalextric, en la que los coches circulan
 fijos a una l�nea de la pista. Lo interesante de este reto es que la f�sica cambiaba de un circuito a otro, por lo que era importante intentar 
 descubrir las caracter�sticas de cada pista antes de empezar. Adem�s, se daba la circunstancia de que en el juego simulado, si la velocidad en la curva
 era demasiado elevada, el coche era expulsado de la pista, de forma que el coche perd�a todas sus posibilidades de ganar la carrera. Uno de los retos
 importantes en este caso, adaptar la velocidad a la m�xima posible sin llegar a salir de la pista. ~\\
 
\begin{table}[]
\centering
\caption{Hello World Open 2014}
\label{hwo-2014}
\begin{tabularx}{\textwidth}{|X|} \hline
�mbito: Internacional \\
Lugar: Online, sitio web de la organizaci�n \\
A�o: 2014 \\
Modalidades: Manejo de una comunidad de agentes rob�ticos \\ 
N�mero de participantes: 2.520 equipos \\
Web: https://2014.helloworldopen.com/ \\ \hline
\end{tabularx}
\end{table}
 

\begin{table}[]
\centering
\caption{Asignaturas por curso y competiciones}
\label{asignaturas-competiciones}
\begin{tabularx}{\textwidth}{|lXl|} \hline
Curso & Asignaturas & Competici�n\\ \hline
2004/2005 & PD, IIA & \\ \hline
2005/2006 & PD, IA, IAeIC &  \\ \hline
2006/2007 & PD, IA, IAeIC &  \\ \hline
2007/2008 & PD, IA, IAeIC &  \\ \hline
2008/2009 & PD, IAeIC, LIA & Cosmobot 2009 \cite{carpio2011} \\ \hline
2009/2010 & PD, IAeIC, LIA & CIAR 2010 \\ \hline
2010/2011 & PD, LIA & ANTS 2011 \cite{carpio2013, carpio2014} \\ \hline
2011/2012 & IAeIC, LIA, MD & FLL 2011/2012 \\ \hline
2012/2013 & IAeIC, RC & FLL 2012/2013 \\ \hline
2013/2014 & RC, MAC & HWO 2014 \\ \hline
\multicolumn{3}{|l|}{PD: Programaci�n declarativa} \\
\multicolumn{3}{|l|}{IA: Inteligencia Artificial} \\ 
\multicolumn{3}{|l|}{IIA: Introducci�n a la Inteligencia Artificial} \\ 
\multicolumn{3}{|l|}{LIA: Laboratorio de Inteligencia Artificial} \\ 
\multicolumn{3}{|l|}{IAeIC: IA e Ingenier�a del Conocimiento} \\ 
\multicolumn{3}{|l|}{MD: Miner�a de datos} \\
\multicolumn{3}{|l|}{MAC: Modelos Avanzados de Computaci�n} \\ 
\multicolumn{3}{|l|}{RC: Representaci�n del Conocimiento} \\ \hline
\end{tabularx}
\end{table}
 
\section{Estructura de la tesis}

A continuaci�n se expone la estructura del resto de cap�tulos: 

En el Cap�tulo \ref{chap:objetivos} (Objetivos) se plantean las preguntas de investigaci�n que motivan el presente trabajo, en el 
Cap�tulo \ref{chap:metodolog\'ia} (Metolog�a)
 se expone el camino seguido para alcanzar los objetivos, en el Cap�tulo \ref{chap:resultados} (Resultados) se destacan los hitos conseguidos, y en 
 el cap�tulo \ref{chap:conclusiones} (Conclusiones) encontramos las reflexiones generales sobre el trabajo de tesis y propuestas de trabajos futuros.}{\myChapter{Introducci{\'o}n}\label{chap:introduction}}

% Los objetivos deber�an ir antes de todo lo dem�s. La introducci�n,
% detr�s de los objetivos - JJ
\ifthenelse{\equal{\value{IncluyeCapitulo}}{2}}{\myChapter{Objetivos}\label{chap:objetivos}

\minitoc\mtcskip
\vfill


\section{Objetivos de esta tesis} 

\lettrine{E}l objetivo de esta tesis es crear una metodolog�a que permita mejorar la experiencia docente en el �mbito de la IA/IC. Esta
motodolog�a mejora uno de los aspectos imprescindibles en el proceso de aprendizaje como es la motivaci�n. A continuaci�n se describen 
los sub-objetivos de la tesis:




\section{Objetivos} 


Los objetivos que esta tesis quiere validar son los siguientes:
\newcommand{\objectiveparadigmSPANISH}{Probar que la inclusi�n de competiciones en el aula puede mejorar la experiencia de aprendizaje de la IA}


\subsection*{Objetivo 1: \objectiveparadigmSPANISH}

Las competiciones en IA pueden favorecer el aprendizade de t�cnicas que tradicionalmente se impart�an de una forma te�rica con baja implicaci�n de los alumnos.


\newcommand{\objectivemethodologySPANISH}{Proponer una metodolog�a que ayude a los profesores responsables de asignaturas relacionadas con la IA a mejorar la experiencia
de aprendizaje de sus alumnos}

\subsection*{Objetivo 2: \objectivemethodologySPANISH} 

Describir los recursos necesarios para desarrollar esta metodolog�a de forma pr�ctica.

\newcommand{\objectiveframeworkSPANISH}{Validar la metodolog�a a trav�s de dos experiencias reales en el aula}
\subsection*{Objetivo 3: \objectiveframeworkSPANISH}


Para validar la metodolog�a se han realizado diferentes experiencias en el aula cuyos resultados han sido publicados en diferentes revistas cient�ficas.

\newcommand{\objectiveresearchSPANISH}{Validaci�n de la metodolog�a a trav�s de la publicaci�n de un art�culo cient�fico con la participaci�n de alumnos} 
\subsection*{Objetivo 4: \objectiveresearchSPANISH}

Como objetivo final de la tesis, se ha realizado un trabajo de investigaci�n por parte de alumnos que han trabajado en este nuevo modelo de aprendizaje.


\section{Estructura de la tesis}

Este cap�tulo ofrece una introducci�n a esta tesis, incluyendo su motivaci�n y las preguntas a abordar.

A continuaci�n se expone la estructura del resto de cap�tulos:

El primer paso de esta tesis ha consistido en el estudio de las t�cnicas de IA/IC utilizadas en el �mbito universitario. Fruto de este trabajo inicial se ha
publicado el primer trabajo cient�fico en revista internacional \cite{Vrivas2003Fuzzy}

El segundo trabajo \cite{Carpio2011Classroom}, el tercer trabajo \cite{CarpioIwann2013} y el cuarto \cite{CarpioJcal2014}.






}{\myChapter{Objetivos}\label{chap:objetivos}}

\ifthenelse{\equal{\value{IncluyeCapitulo}}{2}}{\myChapter{Metodolog\'ia}\label{chap:metodolog\'ia}

\begin{flushright}{\slshape
    "Todo lo que no se da se pierde."} \\ \medskip
    --- {Proverbio indio}
\end{flushright}

\minitoc\mtcskip
\vfill


\section{Metodolog\'ia} 

\lettrine{C}{on} el fin de llevar a cabo este trabajo de investigaci�n, hemos seguido como gu�as metodol�gicas principales las buenas pr�cticas del m�todo cient�fico y, en lo referente a desarrollos, las buenas pr�cticas de la Ingenier�a del Software y la Ingenier�a de los Computadores.  Teniendo esto como marco general, hemos estructurado nuestro trabajo en las siguientes etapas:

\begin{enumerate}
\item	Adquisici�n de conocimientos relacionados con la investigaci�n, el m�todo cient�fico, dise�o de experimentos, an�lisis de datos y resultados en el �mbito de la IA  
\item	B�squeda de competiciones que se ajusten al periodo lectivo 
\item	Breve estudio de las caracter�sticas de la competici�n
\item	Propuesta de la actividad docente de gamificaci�n al alumnado
\item	Dise�o de la experiencia de gamificaci�n en la ense�anza de la IA en ingenier�a con las siguientes etapas:
\begin{enumerate}
\item	Dise�o de las encuestas para evaluar los diferentes elementos 
\item	Planificaci�n de la experiencia docente
\end{enumerate}
\item	Recopilaci�n de datos previos a la experiencia
\item	Puesta en marcha de la actividad docente
\item	Recopilaci�n de datos posteriores a la experiencia
\item	An�lisis de los datos obtenidos
\item	An�lisis de las conclusiones
\item	Publicaci�n en revistas de impacto de las experiencias
\end{enumerate}

La primera fase de la metodolog�a es el conocer el modo de elaborar un art�culo cient�fico en el �mbito de la IA. Para ello, en primer lugar, 
se realiza una revisi�n bibliogr�fica de diferentes t�cnicas, en nuestro caso las Redes Neuronales y concretamente los Mapas Auto-organizativos 
(SOM), los algoritmos gen�ticos y la l�gica borrosa. Esta fase es fundamental, ya que sin estos conocimientos no ser�a posible dise�ar la actividad
 de gamificaci�n en el aula como un experimento cient�fico que nos permita publicar resultados en una revista de prestigio (con clasificaci�n en JCR).
 Como parte de esta primera etapa se elabora un trabajo cient�fico en colaboraci�n con otros compa�eros en el �mbito de la l�gica borrosa y la 
 computaci�n evolutiva \cite{vrivas2003}. ~\\
 
Una vez conocido como aplicar el m�todo cient�fico y con la experiencia de una primera publicaci�n, necesitamos adaptar esos conceptos aprendidos al
 estudio de un experimento en el aula. El m�todo cient�fico como es bien sabido se basa en dos pilares fundamentales: el principio de reproducibilidad
 y de refutabilidad. Es decir, necesitamos describir la experiencia de forma que pueda ser reproducible y adem�s tenemos que publicar el trabajo 
 realizado de forma que la comunidad cient�fica pueda poner en marcha la experiencia y corroborar o no la certeza de lo expuesto. Al trabajar con 
 estudiantes, a diferencia con la experimentaci�n con un conjunto de datos, la forma de reproducir la experiencia nunca producir� resultados 
 exactamente iguales. Por esta raz�n, este tipo de experiencias nos obliga a considerar la reproducibilidad desde un punto de vista no tan estricto
 al que tendr�amos considerando por ejemplo un algoritmo aplicado a un conjunto de datos conocido. Superada esta consideraci�n, y con la fuerza de
 los beneficios observados en los alumnos que participan en competiciones de IA, decidimos dar un tratamiento cient�fico a la experiencia con la
 esperanza de poder publicar los resultados obtenidos en revistas de prestigio. ~\\
 
Para poder poner en marcha una experiencia de gamificaci�n en el aula necesitamos hacer coincidir el periodo lectivo con alguna de las competiciones
 relacionadas con la IA. Esto no siempre es posible, por lo que en algunos casos, como sucedi� en CIAR 2010, los propios profesores toman parte activa 
 organizando una competici�n para que los alumnos puedan participar. Esta opci�n requiere de un gran esfuerzo por parte de los profesores organizadores,
 por lo que no siempre ser� una alternativa adecuada. La otra opci�n m�s factible, requiere conocer competiciones de IA y para ello es importante 
 disponer de una red de contactos con inter�s en el mundo de las competiciones en IA. En nuestro caso ha sido esta red de contactos la que nos ha 
 permitido conocer nuevas iniciativas o reediciones de competiciones anteriores que hemos podido encajar dentro del periodo lectivo de las asignaturas
 de IA impartidas. Es importante destacar en este punto que hay competiciones que se celebran durante varios a�os seguidos y despu�s dejan de 
 celebrarse por algunos a�os (como es caso de AI Challenge) o bien otras nuevas surgen (como el caso de Hello World Open Competition que comienza
 en 2014 y tiene prevista una nueva edici�n en 2016). ~\\
 
Una vez decidido en que competici�n participar, el siguiente paso es hacer un peque�o estudio de la competici�n. Es importante conocer las reglas, 
lenguajes de programaci�n que podemos utilizar, las fases de la competici�n, los plazos de inscripci�n y finalizaci�n, recursos necesarios, 
obligatoriedad o no de desplazarse al lugar del evento, recursos disponibles (programas de ejemplo), foros de discusi�n, etc. Con toda esta informaci�n
 el profesor debe valorar la viabilidad de la puesta en marcha de la actividad dentro del aula. En este sentido, la experiencia puede ayudarnos a 
 decidir la viabilidad o no de la puesta en marcha de la actividad. ~\\
 
A continuaci�n, el profesor plantea a los alumnos la posibilidad de realizar la actividad de gamificaci�n en el aula. En \cite{carpio2014} se pone
 de manifiesto que los resultados son mejores cuando son los alumnos tienen la posibilidad de decidir si participan en la competici�n. De esta forma los
 alumnos aceptan el reto propuesto y lo toman como un proyecto personal. El mismo m�todo se aplic� en \cite{carpio2011} y los resultados en cuanto
 a motivaci�n e implicaci�n fueron muy positivos. ~\\
 
Una vez aceptado el reto por parte de los alumnos, comienza el proceso de planificaci�n de la actividad por parte del profesor. Con la idea de no alterar
 demasiado la marcha habitual del curso en cuanto a sus actividades y sus contenidos, por lo general lo que hemos hecho ha sido concentrar el trabajo
 en un periodo corto. En la experiencia descrita en \cite{carpio2011}, el trabajo se concentra principalmente en la semana previa a la competici�n.
 Durante esta semana, los participantes organizaron reuniones de trabajo que ocuparon casi todo un fin de semana.  De esta forma los estudiantes 
 aprendieron a organizar bien el tiempo, comprobando cu�les son sus l�mites y en qu� momento es mejor hacer un descanso para que las horas de trabajo
 vuelvan a ser productivas. Con este m�todo organizativo conseguimos alterar m�nimamente la planificaci�n establecida para el curso. Solo se introdujeron
 algunas sesiones en las que se inform� de las reglas de la competici�n \cite{carpio2011, carpio2014} y algunas sesiones para realizar algunas tareas
 espec�ficas \cite{carpio2011} como por ejemplo para el dise�o de la placa de circuito impreso PCB o para la fabricaci�n y montaje de la placa de
 control del robot m�vil. En el caso de ANTS 2011 se organizaron un par de sesiones para mostrar c�mo crear un  agente rob�tico b�sico a partir de los
 ficheros proporcionados por la organizaci�n y en el fin de semana anterior se organiz� una sesi�n de trabajo que ocup� casi todo el fin de semana. ~\\
 
Una vez introducido el problema a resolver y puesto que algunos de los conceptos impartidos en la asignatura se pod�an aplicar al agente rob�tico,
 los alumnos encontraban r�pidamente la utilidad de lo explicado, como por ejemplo en el caso de encontrar un camino m�nimo hacia la comida o el
 hormiguero enemigo en ANTS 2011 \cite{carpio2013, carpio2014} con el algoritmo A*. En cursos anteriores, era necesario introducir ejemplos, no
 siempre cercanos a los intereses de los alumnos, que hac�a dif�cil que ellos pudiesen comprobar la utilidad real de la t�cnica que se trata de 
 ense�ar, lo que derivaba en falta de inter�s y motivaci�n. Sin embargo, al tratar de resolver un reto planteado los alumnos encuentran r�pidamente
 la aplicaci�n de lo que est�n aprendiendo y al ver la utilidad adquieren los conocimientos de una forma m�s r�pida y posiblemente m�s duradera.~\\
 
Podemos resumir la metodolog�a de la siguiente forma:
\begin{itemize}
\item	M�nima alteraci�n de la estructura habitual de la asignatura (contenidos y planificaci�n temporal)
\item	Concentraci�n del trabajo en sesiones de fin de semana
\item	Aplicaci�n de los conceptos aprendidos en el desarrollo del agente para la competici�n (A*, algoritmos evolutivos, l�gica borrosa, etc.)
\item	Recopilaci�n de datos de las encuestas de opini�n de los alumnos
\item	An�lisis de los resultados
\end{itemize}}{\myChapter{Metodolog\'ia}\label{chap:metodologia}}

\ifthenelse{\equal{\value{IncluyeCapitulo}}{2}}{\myChapter{Discusi\'on y Resultados}\label{chap:resultados}

\minitoc\mtcskip
\vfill


4. Discusi�n y resultados
 - Aqu� �bastar�a? con traducir los resultados de cada uno de los art�culos anteriores




\section{Discusi\'on y resultados} 

\lettrine{D}iscusi�n y resultados ...





}{\myChapter{Resultados}\label{chap:resultados}}

\ifthenelse{\equal{\value{IncluyeCapitulo}}{2}}{\myChapter{Conclusiones}\label{chap:conclusiones}

\minitoc\mtcskip
\vfill


5. Conclusiones
 - ???


\section{Conclusiones} 

\lettrine{E}l objetivo de esta tesis es crear una metodolog�a para adaptar Algoritmos Evolutivos 
(AEs) a entornos din�micos, heterog�neos y basados en est�ndares. Esta metodolog�a propone el uso de 
Arquitectura Orientada a Servicios (AOS) como un nuevo paradigma para desarrollar AEs. Para validar
 esta metodolog�a se usar� para crear un {\em framework} que use todas las ventajas de este paradigma
 (enlace din{\'a}mico y distribuci{\'o}n y publicaci{\'o}n de interfaces utilizando est�ndares). Finalmente,
 esta metodolog�a se usar� para crear Algoritmos Evolutivos Orientados a Servicios (AEOS) en diferentes escenarios,
 donde se reflejar�n estas ventajas.

}{\myChapter{Conclusiones}\label{chap:conclusiones}}

%-----------------------------
% -- \myPart{Materials and Methods}\label{part:metodoYmateriales}
% -- \ifthenelse{\equal{\value{IncluyeCapitulo}}{2}}{\include{Chapters/03-soa}}{\myChapter{Service Oriented Architecture: technologies and restrictions for designing services for EAs}\label{chap:soa}}
% -- \ifthenelse{\equal{\value{IncluyeCapitulo}}{2}}{\include{Chapters/04-soaea}}{\myChapter{A methodology for developing services for EAs}\label{chap:soaea}}
%-----------------------------
% -- \myPart{Experimental Results}\label{part:resultadosExperimentales}
% -- \ifthenelse{\equal{\value{IncluyeCapitulo}}{2}}{\include{Chapters/05-osgiliath}}{\myChapter{Implementation of SOA-EA}\label{chap:osgiliath}}
% -- \ifthenelse{\equal{\value{IncluyeCapitulo}}{2}}{\include{Chapters/06-adaptive}}{\myChapter{Parameter adaptation in heterogeneous machines}\label{chap:adaptive}}
%\ifthenelse{\equal{\value{IncluyeCapitulo}}{2}}{\include{Chapters/07-art}}{\myChapter{HSV and RGB comparison in Evolutionary Art}\label{chap:art}}
% -- \ifthenelse{\equal{\value{IncluyeCapitulo}}{2}}{\include{Chapters/08-rts}}{\myChapter{Generation of bots for RTS games using Genetic Programming}\label{chap:rts}}

%-----------------------------
% -- \myPart{Conclusions}\label{part:discusionYconclusiones}
% -- \ifthenelse{\equal{\value{IncluyeCapitulo}}{2}}{\include{Chapters/09-conclusions}}{\myChapter{Conclusions and future work}\label{chap:conclusions}}
% -- \ifthenelse{\equal{\value{IncluyeCapitulo}}{2}}{\include{Chapters/09-conclusions-SPANISH}}{\myChapter{Conclusiones y trabajo futuro}\label{chap:conclusionsSPANISH}}
%\ifthenelse{\equal{\value{IncluyeCapitulo}}{12}}{\include{Chapters/conclusiones}}{\myChapter{Conclusiones}\label{chap:conclusiones}}

}


% *******************************************************
% Publicaciones
% *******************************************************
\myPart{Publicaciones}
% \include{Chapters/appendixosgi}
% \include{Chapters/appendixosgiliath}

% JCC PDFs
% ---- Artículo 1

\section{Evolving two-dimensional fuzzy systems} 

A continuaci\'on se detallan los datos del art\'iculo publicado relacionado con esta secci\'on de la disertaci\'on.\\
\\
T\'itulo: \textbf{Evolving two-dimensional fuzzy systems}\\
Revista: \textbf{Fuzzy Sets and Systems}, 2003; Factor de impacto: \textbf{0,577};\\
Autores: V\'ictor M. Rivas, J.J. Merelo, I. Rojas, G. Romero, P.A. Castillo,  \textbf{J. Carpio}\\
\\
Relevancia de la revista:\\
\\
\begin{tabular}{ l c c c }
 \hline
  \fontsize{10}{12} \selectfont \specialcell{Nombre de la categor\'ia} & \fontsize{10}{12} \selectfont \specialcell{Revistas en\\la categor\'ia} & \fontsize{10}{12} \selectfont  \specialcell{Posici\'on en\\la categor\'ia} & \specialcell{Cuartil en\\la categor\'ia} \\
 \hline
  \fontsize{10}{12} \selectfont \specialcell{COMPUTER SCIENCE,\\ THEORY} & 70 & 44 & Q3\\
  \fontsize{10}{12} \selectfont \specialcell{MATHEMATICS,\\ APPLIED} & 153 & 85 & Q3 \\
  \fontsize{10}{12} \selectfont \specialcell{STATISTICS\\ \& PROBABILITY} & 75 & 42 & Q3 \\
   \hline
\end{tabular}




\setboolean{@twoside}{false}
%-- Artículos comentados para realizar la compilación más rápido
-- \includepdf[pages={-},scale=1.0,pagecommand={},offset=0 0,
-- addtolist={1, table, {Evolving two-dimensional fuzzy systems}, tab:evolving}]{pdf/evolving.pdf}
 
 % Página en blanco
 \newpage\null\thispagestyle{empty}\newpage
 
% ---- Artículo 2 

\section{From Classroom to Mobile Robots Competition Arena: ... } 

A continuaci\'on se detallan los datos del art\'iculo publicado relacionado con esta secci\'on de la disertaci\'on.\\
~\\
T\'itulo: \textbf{From Classroom to Mobile Robots Competition Arena: An Experience on Artificial Intelligence Teaching}\\
Revista: \textbf{ The International journal of engineering education}, 2011; Factor de impacto: \textbf{0,418};\\
Autores: \textbf{Jos\'e Carpio Ca\~nada}, T. J. Mateo Sanguino, S. Alcocer, A. Borrego, A. Isidro, A. Palanco, J.M. Rodr\'iguez\\
~\\
Relevancia de la revista:\\
~\\
\begin{tabular}{ l c c c }
 \hline
  \fontsize{10}{12} \selectfont \specialcell{Nombre de la categor\'ia} & \fontsize{10}{12} \selectfont \specialcell{Revistas en\\la categor\'ia} & \fontsize{10}{12} \selectfont  \specialcell{Posici\'on en\\la categor\'ia} & \specialcell{Cuartil en\\la categor\'ia} \\
 \hline
  \fontsize{10}{12} \selectfont \specialcell{EDUCATION,\\ SCIENTIFIC DISCIPLINES} & 33 & 24 & Q4\\
  \fontsize{10}{12} \selectfont \specialcell{ENGINEERING,\\ MULTIDISCIPLINARY} & 90 & 60 & Q3 \\
   \hline
\end{tabular}




\setboolean{@twoside}{false}
-- \includepdf[pages={-},scale=1.0,pagecommand={},offset=0 0,
-- addtolist={1, table, {From Classroom to Mobile Robots Competition Arena: An Experience on Artificial Intelligence Teaching}, tab:evolving}]{pdf/ijee2469ns.pdf} 
 
 % Página en blanco
 \newpage\null\thispagestyle{empty}\newpage
 
% ---- Artículo 3 

\section{Evolving the Strategies of Agents for the ANTS Game} 

A continuaci\'on se detallan los datos del art\'iculo publicado relacionado con esta secci\'on de la disertaci\'on.\\
\\
T\'itulo: \textbf{Evolving the Strategies of Agents for the ANTS Game}\\
Congreso: \textbf{International Work-Conference on Artificial Neural Networks (IWANN)}, 2013\\
Autores: \textbf{Jos\'e Carpio}, Pablo Garc\'ia-S\'anchez, Antonio Miguel Mora, Juan Juli\'an Merelo Guerv\'0s, Jes\'us Caraballo, Ferm\'in Vaz, Carlos Cotta\\
\\
Relevancia del congreso atendiendo al \"Computer Science Conference Ranking\"\\
\\
\begin{tabular}{ l c c }
 \hline
  \fontsize{10}{12} \selectfont Nombre del \'Area & \fontsize{10}{12} \selectfont  \specialcell{Posici\'on en\\la categor\'ia} \\
 \hline
  \fontsize{10}{12} \selectfont \specialcell{Artificial Intelligence\\ and Related Subjects} & \fontsize{10}{12} \selectfont Rank3\\
   \hline
\end{tabular}




\setboolean{@twoside}{false}
-- \includepdf[pages={-},scale=1.0,pagecommand={},offset=0 0,
-- addtolist={1, table, {Evolving the Strategies of Agents for the ANTS Game}, tab:evolving}]{pdf/ants_iwann2013.pdf}
 
 % Página en blanco
 \newpage\null\thispagestyle{empty}\newpage
 
% ---- Artículo 4 

\section{Open classroom: enhancing student achievement on artificial ...} 

T\'itulo: \textbf{Open classroom: enhancing student achievement on artificial intelligence through an international online competition}\\
Revista: \textbf{Journal of Computer Assisted Learning}, 2014; Factor de impacto: \textbf{1,023};\\
Autores: J. Carpio Ca\~nada, T.J. Mateo Sanguino, J.J. Merelo Guerv\'os, V.M. Rivas Santos\\
~\\
Relevancia de la revista:\\
~\\
\begin{tabular}{ l c c c }
 \hline
  \fontsize{10}{12} \selectfont \specialcell{Nombre de la categor\'ia} & \fontsize{10}{12} \selectfont \specialcell{Revistas en\\la categor\'ia} & \fontsize{10}{12} \selectfont  \specialcell{Posici\'on en\\la categor\'ia} & \specialcell{Cuartil en\\la categor\'ia} \\
 \hline
  \fontsize{10}{12} \selectfont \specialcell{EDUCATION\\ \& EDUCATIONAL RESEARCH} & 219 & 61 & Q2\\
   \hline
\end{tabular}




\setboolean{@twoside}{false}
-- \includepdf[pages={-},scale=1.0,pagecommand={},offset=0 0,
--  addtolist={1, table, {Open classroom: enhancing student achievement on artificial intelligence through an international online competition}, tab:evolving}]{pdf/jcal12075.pdf}
 
  % Página en blanco
 \newpage\null\thispagestyle{empty}\newpage

% *******************************************************
% Backmatter
% *******************************************************
\appendix
\myPart{Ap\'endice}
% \include{Chapters/appendixosgi}
% \include{Chapters/appendixosgiliath}


%\include{Chapters/apendiceC}
%********************************************************************
% Other Stuff in the Back
%*******************************************************
\cleardoublepage
\include{FrontBackmatter/Bibliography}
%\cleardoublepage
%\include{FrontBackmatter/Declaration}
\cleardoublepage
%La contraportada tiene que estar en página par!!
% \include{FrontBackmatter/DirtyBackpage}

% ********************************************************************
% Game Over: Restart, Restore or Quit?
%*******************************************************
\listoftodos[Notes]
\end{document}
% ********************************************************************
