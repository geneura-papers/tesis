%*******************************************************
% Acknowledgments
%*******************************************************
\pdfbookmark[1]{Agradecimientos}{agradecimientos}

\begin{flushright}
 
\end{flushright}



\bigskip

\begingroup
\let\clearpage\relax
\let\cleardoublepage\relax
\let\cleardoublepage\relax
\chapter*{Agradecimientos}
Mis agradecimientos van en primer lugar a mi familia Manuel Jos�, Matilde, Manuel �ngel, Fernando, Lolita y Elisa Matilde que son las ramitas que dan fuerza 
y sentido a la vida.
A JJ y V�ctor, por su paciencia, por dejar que me equivocase tantas veces y sobre todo por haber sido capaces de darme la �ltima oportunidad despu�s
 de tantas anteriores.   
A Nicol�s, Carmen, Alicia, Nico y Eloy que ha sido mi segunda familia durante las estancias en el CITIC por su apoyo, sus risas matutinas, sus consejos
 y sus maravillosas cenas.
A mis compa�eros de Geneura que me acompa�aron en este largo viaje JJ, V�ctor, Pedro, Gustavo, Pablo, Juanlu, Antonio Mora, Maribel, Paloma, Antares a
 los que espero poder devolver alg�n d�a todo lo que me han dado. A mis compa�eros de Huelva, Nacho, Miguel �ngel, Fran, Gonzalo, Manuel, Manuel Emilio, 
 Manolo Vasallo, Teresa, Jos� Luis �lvarez, Jos� Luis Arjona, Paco M�rquez, Patricia, I�aki, Nieves, Antonio Peregr�n, Manolo Redondo, Javier Aroba por
 las maravillosas charlas sobre temas cient�ficos y no cient�ficos. A Jos� Manuel And�jar por apostar por mi desde el principio y por dejarme meter la
 pata. A los compa�eros de Simple LAB Juan Diego, Manuel, Miguel �ngel, Juan Manuel Enrique y Antonio por tanta energ�a e ilusi�n puesta en los nuevos
 proyectos. A la gente de CRM y en especial a Antonio Palanco y su familia por estar siempre dispuesto a echar un cable para lo que sea. 
A Carmen, Cristina, Fernando y todos los que colaboraron en hacer que la FLL se celebrase en Huelva por primera y segunda vez.   
A los compa�eros y personal del CITIC por hacer de ese espacio un hogar para muchos investigadores.  Al Dr. Tom�s Mateo Sanguino sin cuyo trabajo y tes�n
 no hubiese sido posible esta tesis. Al Dr. Arturo Aquino, por sus sabios consejos, su apoyo y su amistad. A Jos� Enrique Cano por su amistad y por
 ense�arme a dedicarme con pasi�n a descubrir, sea en el �mbito que sea. A mis amigos Nicol�s, Nico, Alejandro, Javi Aguayo, Jes�s, Alex, Dani y al 
 resto de la familia de Tarifa, Andy, Samuel, Silvia, Amparo, Lidia, Marcos Benavent, Felipe, Rafa, David, Ant�n, Paqui, Juan Torres por su apoyo tanto en lo 
 profesional como en lo personal.
A mis alumnos, los que participaron en competiciones y art�culos cient�ficos y a los que no, por ser una pieza tan importante en el d�a a d�a y a quienes
 en especial va dedicado este trabajo.
"Si est�s leyendo esto cerca de m� y no te he nombrado es que soy un desastre y merezco que me des una colleja ... Gracias a todos. Por cierto, si est�s
 leyendo esto y no te conozco quiere decir que te interesa esta tesis, as� que te la dedico a ti tambi�n, qu� demonios." Gracias Pablo por la frase, no 
 he visto una mejor forma de terminar esta secci�n.
 

\endgroup



