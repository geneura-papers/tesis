%*******************************************************
% Abstract
%*******************************************************
%\renewcommand{\abstractname}{Abstract}
% \pdfbookmark[1]{Abstract}{Abstract}
\begingroup
\let\clearpage\relax
\let\cleardoublepage\relax
\let\cleardoublepage\relax



\pdfbookmark[1]{Resumen}{Resumen}
% \chapter*{Abstract}


\chapter*{Resumen}

% \section{Resumen}

Los m�todos tradicionales de ensa�anza pueden influenciar de forma negativa en la motivaci�n y en las expectativas de los estudiantes,
debido entre otros a la comunicaci�n unidireccional, las metodolog�as
r�gidas o los enfoques orientados a resultados, provocando una reducci�n del rendimiento acad�mico. Con el objetivo de hacer que el proceso 
de aprendizaje sea motivante esta tesis presenta una metodolog�a que
permite mejorar la experiencia de aprendizaje de alumnos y profesores. Como aplicaci�n pr�ctica de la metodolog�a propuesta, se han llevado a cabo varias experiencias
reales en asignaturas clasicas de Ingeligencia Artificial. En estas asignaturas se han sustitu�do algunas sesiones por la participaci�n
en competiciones nacionales e internacionales de juegos basados en Ingeligencia Artificial que ten�an como objetivo la realizaci�n de un
 agente capaz de competir contra otros adversarios. Se han analizado entre otros elementos el ranking en la competici�n,
la opini�n de los estudiantes o el progreso acad�mico con el fin de evaluar la metodolog�a empleada. Hemos comprobado como la experiencia
educacional mojora la percepci�n global de los estudiantes, mejorando incluso sus resultados acad�micos y sus habilidades personales.
 Como conclusi�n, esta metodolog�a nos ha permitido comprobar que el proceso es m�s importante que el resultado y que es posible adaptarla
 a diferentes escenarios de aprendizaje dentro de una instituci�n acad�mica.
 ~\\


\vfill


\endgroup

\vfill