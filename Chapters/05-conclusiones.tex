\myChapter{Conclusiones}\label{chap:conclusiones}


\begin{flushright}{\slshape
    "La ciencia no es sino una perversi�n de s� misma a menos que tenga como objetivo final 
	el mejoramiento de la humanidad".} \\ \medskip
    --- {Nikola Tesla}
\end{flushright}
 
\minitoc\mtcskip
\vfill


\section{Conclusiones} 

\lettrine{T}{ras} analizar los diferentes objetivos propuestos y los resultados obtenidos, podemos concluir que la metodolog�a propuesta para
 el uso de la gamificaci�n en el aprendizaje de la IA es adecuada y que obtiene mejoras significativas en aspectos como la motivaci�n y 
 la mejora de los conocimientos adquiridos. Que es posible introducir de forma satisfactoria este tipo de experiencias dentro de los planes
 de estudios tradicionales y que es posible compatibilizar el enfoque m�s tradicional de la ense�anza combin�ndolo con juegos competitivos
 de forma que mejore la experiencia de aprendizaje de los estudiantes en el �mbito de la IA. ~\\
 
Adem�s de las experiencias descritas en las publicaciones presentadas en esta tesis, se han desarrollado nuevas experiencias que pretenden 
seguir profundizando en el uso de la gamificaci�n en el aula. 

Nuestras l�neas de trabajos futuros est�n orientadas a

\begin{itemize}
\item El uso de la gamificaci�n con el fin de realizar de forma colaborativa tareas complejas.

\item Ampliar el �mbito de futuros estudios de forma que se puedan implicar diferentes Universidades y obtener as� datos m�s precisos sobre
 la influencia de este tipo de experiencias.
 
\item Analizar el posible uso de la gamificaci�n para en otros �mbitos distintos a la educaci�n como pueden ser el emprendimiento. En esta
 l�nea, para el curso 2015/2016 ya ha comenzado la organizaci�n de una competici�n de veh�culos el�ctricos solares denominada Desaf�o Solar 
 Costa de la Luz 2016 (DSCL 2016)que implicar� a estudiantes de Ingenier�a y de ense�anzas medias.
 \end{itemize}
 
 
