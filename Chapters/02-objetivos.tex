\myChapter{Objetivos}\label{chap:objetivos}

\minitoc\mtcskip
\vfill


\section{Objetivos de esta tesis} 

\lettrine{E}l objetivo de esta tesis es crear una metodolog�a que permita mejorar la experiencia experiencia docente en el �mbito de la IA/IC. 



\section{Objetivos} 


Los objetivos que esta tesis quiere validar son los siguientes:
\newcommand{\objectiveparadigmSPANISH}{Probar que la inclusi�n de competiciones en el aula puede mejorar la experiencia de aprendizaje de la IA}


\subsection*{Objetivo 1: \objectiveparadigmSPANISH}

Las competiciones en IA pueden favorecer el aprendizade de t�cnicas que tradicionalmente se impart�an de una forma te�rica con baja implicaci�n de los alumnos.


\newcommand{\objectivemethodologySPANISH}{Proponer una metodolog�a que ayude a los profesores responsables de asignaturas relacionadas con la IA a mejorar la experiencia
de aprendizaje de sus alumnos}

\subsection*{Objetivo 2: \objectivemethodologySPANISH} 

Describir los recursos necesarios para desarrollar esta metodolog�a de forma pr�ctica.

\newcommand{\objectiveframeworkSPANISH}{Validar la metodolog�a a trav�s de dos experiencias reales en el aula}
\subsection*{Objetivo 3: \objectiveframeworkSPANISH}


Para validar la metodolog�a se han realizado diferentes experiencias en el aula cuyos resultados han sido publicados en diferentes revistas cient�ficas.

\newcommand{\objectiveresearchSPANISH}{Probar que una implementaci�n basada en SOA de algoritmos evolutivos distribuidos, din�micos y basados en est�ndares
  tiene la capacidad de resolver de forma eficiente distintos problemas} %FERGU: OJO!! A�adir el problema de GP?


\subsection*{Objetivo 4: \objectiveresearchSPANISH}
Como objetivo final de la tesis, se ha realizado un trabajo de investigaci�n por parte de alumnos que han trabajado en este nuevo modelo de aprendizaje.


\section{Estructura de la tesis}

Este cap�tulo ofrece una introducci�n a esta tesis, incluyendo su motivaci�n y las preguntas a abordar.

A continuaci�n se expone la estructura del resto de cap�tulos:

El primer paso de esta tesis ha consistido en el estudio de las t�cnicas de IA/IC utilizadas en el �mbito universitario. Fruto de este trabajo inicial se ha
publicado el primer trabajo cient�fico en revista internacional \cite{Vrivas2003Fuzzy}

El segundo trabajo \cite{Carpio2011Classroom}, el tercer trabajo \cite{CarpioIwann2013} y el cuarto \cite{CarpioJcal2014}.






