\myChapter{Objetivos}\label{chap:objetivos}

\begin{flushright}{\slshape
    "A la mente del principiante se le presentan muchas posibilidades; a la del experto, pocas."} \\ \medskip
    --- {Mente Zen, mente de principiante. Shunryu Suzuki}
\end{flushright}

\minitoc\mtcskip
\vfill


\section{Objetivos de esta tesis} 

\lettrine{E}{l} objetivo de esta tesis es aportar una metodolog�a que permita mejorar la experiencia docente en el �mbito de la Inteligencia Artificial. Esta
motodolog�a mejora uno de los aspectos imprescindibles en el proceso de aprendizaje como es la motivaci�n. A continuaci�n se describen 
los sub-objetivos de la tesis:




\section{Objetivos} 


Los objetivos que esta tesis quiere validar son los siguientes:
\newcommand{\objectiveparadigmSPANISH}{Probar que la inclusi�n de competiciones en el aula puede mejorar la experiencia de aprendizaje de la IA}


\subsection*{Objetivo 1: \objectiveparadigmSPANISH}

Las competiciones en IA pueden favorecer el aprendizade de t�cnicas que tradicionalmente se impart�an de una forma te�rica con baja implicaci�n de los alumnos.


\newcommand{\objectivemethodologySPANISH}{Proponer una metodolog�a que ayude a los profesores responsables de asignaturas relacionadas con la IA a mejorar la experiencia
de aprendizaje de sus alumnos}

\subsection*{Objetivo 2: \objectivemethodologySPANISH} 

Describir los recursos necesarios para desarrollar esta metodolog�a de forma pr�ctica.

\newcommand{\objectiveframeworkSPANISH}{Validar la metodolog�a a trav�s de dos experiencias reales en el aula}
\subsection*{Objetivo 3: \objectiveframeworkSPANISH}


Para validar la metodolog�a se han realizado diferentes experiencias en el aula cuyos resultados han sido publicados en diferentes revistas cient�ficas.

\newcommand{\objectiveresearchSPANISH}{Validaci�n de la metodolog�a a trav�s de la publicaci�n de un art�culo cient�fico con la participaci�n de alumnos} 
\subsection*{Objetivo 4: \objectiveresearchSPANISH}

Como objetivo final de la tesis, se ha realizado un trabajo de investigaci�n por parte de alumnos que han trabajado en este nuevo modelo de aprendizaje.
