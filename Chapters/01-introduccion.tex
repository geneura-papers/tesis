\myChapter{Introducci{\'o}n}\label{chap:introduccion}

\minitoc\mtcskip
\vfill


% El camino que ahora termina con la presentaci�n de esta tesis doctoral comienza con los cursos de doctorado del programa titulado
%  ``Ingenier�a de Computadores: Perspectivas y Aplicaciones�� de la Universidad de Granada en el curso acad�mico 2000/2001. Un comienzo
% que lleg� de la mano del Catedr�tico Juan Juli�n Merelo Guerv�s y el grupo Geneura al que tambi�n pertenece el Dr. V�ctor Manuel
% Rivas Santos co-director de la presente tesis doctoral. 

Esta tesis parte de un estudio inicial de los aspectos fundamentales de algunas Inteligencia Artificial y m�s concretamente la Inteligencia Computacional.


\section{Inteligencia Artificial vs Inteligencia Computacional}



\section{Metodolog�a}

Primera etapa (2000/2005):\\
Estudio de los fundamentos de la IA / IC:\\
\begin{itemize}
\item Algoritmos gen�ticos (Art�culo  Evolving two-dimensional Fuzzy systems)
\item Redes neuronales (Trabajos sobre SOM)
\item L�gica difusa. (Art�culo: Evolving two-dimensional Fuzzy systems)
 \end{itemize}
 
Segunda etapa (2005/2014):
\begin{itemize}
\item Aprendizaje de la docencia en el �mbito de la IA/IC
\item Estudio de la problem�tica relacionada con la ense�anza y el aprendizaje de la IA/IC
\item Detecci�n de algunos aspectos que dificultan el aprendizade de la IA/IC
\item Dise�o de la primera experiencia did�ctica basada en la participaci�n en una competici�n nacional en el �mbito de la IA/IC
\item Publicaci�n de los resultados de esta primera experiencia: Art�culo 2009 IJEE, From Classroom to Mobile Robots Competition Arena...
\item Dise�o de una segunda experiencia did�ctica basada en la participaci�n en una competici�n internacional en el �mbito de la IA/IC 
\item Publicaci�n de los resultados de esta segunda experiencia: Art�culo 2013 JCAL, Open classroom: ...
\item Dise�o de una tercera experiencia investigadora basada en la participaci�n en una competici�n internacional en el �mbito de la IA/IC 
\item Publicaci�n de los resultados de esta tercera experiencia: Art�culo IWANN 2013 en el que alumnos participan en la elaboraci�n de un art�culo cient�fico.
\end{itemize}

 