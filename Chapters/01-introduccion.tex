\myChapter{Introducci{\'o}n}\label{chap:introduccion}

\minitoc\mtcskip
\vfill


% El camino que ahora termina con la presentaci�n de esta tesis doctoral comienza con los cursos de doctorado del programa titulado
%  ``Ingenier�a de Computadores: Perspectivas y Aplicaciones�� de la Universidad de Granada en el curso acad�mico 2000/2001. Un comienzo
% que lleg� de la mano del Catedr�tico Juan Juli�n Merelo Guerv�s y el grupo Geneura al que tambi�n pertenece el Dr. V�ctor Manuel
% Rivas Santos co-director de la presente tesis doctoral. 

Esta tesis parte de un estudio inicial de los aspectos fundamentales de algunas Inteligencia Artificial y m�s concretamente la Inteligencia Computacional.


\section{Inteligencia Artificial vs Inteligencia Computacional}


\lettrine{C}{omputational} intelligence (CI) is a set of nature-inspired computational methodologies and approaches to address complex real-world problems to which traditional approaches, i.e., first principles modeling or explicit statistical modeling, are ineffective or infeasible. Many such real-life problems are not considered to be well-posed problems mathematically, but nature provides many counterexamples of biological systems exhibiting the required function, practically. For instance, the human body has about 200 joints (degrees of freedom), but humans have little problem in executing a target movement of the hand, specified in just three Cartesian[1] dimensions. Even if the torso were mechanically fixed, there is an excess of 7:3 parameters to be controlled for natural arm movement. Traditional models also often fail to handle uncertainty, noise and the presence of an ever-changing context. Computational Intelligence provides solutions for such[2] and other complicated problems and inverse problems. It primarily includes artificial neural networks,[3] evolutionary computation[4] and fuzzy logic.[5][6] In addition, CI also embraces biologically inspired algorithms such as swarm intelligence[7] and artificial immune systems, which can be seen as a part of evolutionary computation, and includes broader fields such as image processing,[8] data mining,[9] and natural language processing.[10] Furthermore other formalisms: Dempster�Shafer theory, chaos theory and many-valued logic are used in the construction of computational models.

The characteristic of "intelligence" is usually attributed to humans. More recently, many products and items also claim to be "intelligent". Intelligence is directly linked to the reasoning and decision making. Fuzzy logic was introduced in 1965 as a tool to formalise and represent the reasoning process and fuzzy logic systems which are based on fuzzy logic possess many characteristics attributed to intelligence. Fuzzy logic deals effectively with uncertainty that is common for human reasoning, perception and inference and, contrary to some misconceptions, has a very formal and strict mathematical backbone ('is quite deterministic in itself yet allowing uncertainties to be effectively represented and manipulated by it', so to speak). Neural networks, introduced in 1940s (further developed in 1980s) mimic the human brain and represent a computational mechanism based on a simplified mathematical model of the perceptrons (neurons) and signals that they process. Evolutionary computation, introduced in the 1970s and more popular since the 1990s mimics the population-based sexual evolution through reproduction of generations. It also mimics genetics in so called genetic algorithms.


\section{Metodolog�a}

Primera etapa (2000/2005):\\
Estudio de los fundamentos de la IA / IC:\\
\begin{itemize}
\item Algoritmos gen�ticos (Art�culo  Evolving two-dimensional Fuzzy systems)
\item Redes neuronales (Trabajos sobre SOM)
\item L�gica difusa. (Art�culo: Evolving two-dimensional Fuzzy systems)
 \end{itemize}
 
Segunda etapa (2005/2014):
\begin{itemize}
\item Aprendizaje de la docencia en el �mbito de la IA/IC
\item Estudio de la problem�tica relacionada con la ense�anza y el aprendizaje de la IA/IC
\item Detecci�n de algunos aspectos que dificultan el aprendizade de la IA/IC
\item Dise�o de la primera experiencia did�ctica basada en la participaci�n en una competici�n nacional en el �mbito de la IA/IC
\item Publicaci�n de los resultados de esta primera experiencia: Art�culo 2009 IJEE, From Classroom to Mobile Robots Competition Arena...
\item Dise�o de una segunda experiencia did�ctica basada en la participaci�n en una competici�n internacional en el �mbito de la IA/IC 
\item Publicaci�n de los resultados de esta segunda experiencia: Art�culo 2013 JCAL, Open classroom: ...
\item Dise�o de una tercera experiencia investigadora basada en la participaci�n en una competici�n internacional en el �mbito de la IA/IC 
\item Publicaci�n de los resultados de esta tercera experiencia: Art�culo IWANN 2013 en el que alumnos participan en la elaboraci�n de un art�culo cient�fico.
\end{itemize}

 